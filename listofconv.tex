%%% Local Variables:
%%% mode: latex
%%% TeX-master: "../doktorarbeit"
%%% End:
\chapter{List of conventions}
Here we list the mathematical and physical 
conventions that are used in this thesis.

\paragraph{Indices:} When Greek characters are used for indices it is understood that the indices runs from
zero to three. When Latin characters are used the indices runs from one to three.   
We use the Einstein summation convention, which means that repeated indexes
are implicitly summed over. For example, in the expression 
\begin{equation}
f_\mu = x^i x_i y_\mu, \nonumber
\end{equation} one should sum over $i$ from one to three and $\mu = (0,1,2,3)$     
\paragraph{Kronecker-delta:} The symbol $\delta^{ij}$ represents the Kronecker-delta,
which is defined as follows
\begin{equation}
\delta^{ij} = \begin{cases} \nonumber
1 & \text{ if } i=j \\ \nonumber
0 & \text{ if } i \neq j \nonumber 
\end{cases}
\end{equation}
\paragraph{Flat space metric:} we use the following signature for the metric tensor of flat space
\begin{equation}
\munu{\eta} = 
\begin{pmatrix}
-1 & 0 & 0 & 0\\ 
 0& 1 & 0 & 0\\ 
0 & 0 & 1 & 0\\ 
0 & 0 & 0 & 1
\end{pmatrix}. \nonumber 
\end{equation}      
 

