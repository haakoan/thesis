%%% Local Variables:
%%% mode: latex
%%% TeX-master: "../../doktorarbeit"
%%% End:
\chapter{Summary and conlusion}
In this thesis we have studied the GW signals predicted
by seven three-dimensional core-collapse supernovae simulations, with state of the art 
neutrino transport. The main focus has been to identify the fingerprints in the signal
from hydrodynamical instabilites operating behind the stalled shock front, in the time between the
recoil of the inner core and the onset of shock revival. However, since two of the simulations
result in sucesful explisons, we have also studied the evolution of the GW emission
for the first few hundred miliseconds after shock revival. Two of the seven models, one of the two
yielding a sucessfull explison, were based on rotating progenitors. The models cover both the
regime where the post-shock region is dominated by neutrino driven convection, and the regime where
the so called standing accretion shock instability dominates the post-shock flow.

The GW emission from the models is dominated by two distinct components. In all models
GWs are emitted at frequencies greater than 250 Hz, what we in this thesis refer to as 
high-frequency emission. The peak frequency, the frequency at which the maximum amount of radiation 
is emitted, of this emission component increases as the simulations progress forward in time. 
This kind of emission had privously been seen in 2D simulations \citep{marek_08,murphy_09,mueller_13}
and have been connected to exicatation of g-mode ossilations in the convectively stable outer layers of the PNS.
In 2D, the main source of these g-modes was indentified to be downflows from the post-shock layer impinging onto the PNS surface.
The overshooting of convective plumes from the convectivly unstable layer within the PNS was found to be a second, but subdominant, exication
mechanisem. PNS convection was only after the onset of shock revival, because of a significant drop in the accreation rate, found to 
be dominating.  The typical angular frequency of such g-modes roughly given by the Brunt-V\"{a}is\"{a}l\"{a}-frequency
the convectively stable outer layers of the PNS. \citet{mueller_13} investigated the dependence
of the Brunt-V\"{a}is\"{a}l\"{a}-frequency on the mass, radius, and surface temperature and explain the secular
increase of the frequency during the contraction of the PNS.

We found, contrary to the results from 2D \citep{marek_08,murphy_09,mueller_13}, that most of the high-frequency emission is 
generated by PNS convection and that downflows from outside of the PNS only player a minor role. The reason why excatation 
of g-modes by downflows is less predominant in 3D is twofold. Firstly, in 2D energy cascades towards larger and larger scales, unlike
in 3D where energy cascedes towards smaller and smaller scales. This means that 2D simulations more easily develop large-scale, and
high-velocity downflows in the post-shock layer, that more effiecently pertrub the PNS surface. The second reason is connected to the first, since
energy is artifically consentrated at large and intermitten scales, the typical timescales of variations in the mass motions around 
the PNS surface are shorter in 2D than in 3D. Consequently, the spectrum of the forcing overlaps better with the natural frequency
of surface g-modes in 2D, better than in 3D.
In the models where a strong spiral SASI mode dominates the post-shock flow we find perdicular strong high freqyency emission. These models
also emits high-frequency GWs in a wider frequency range. The spiral SASI mode leads to choherent and strong downflows that are able to
strongly perturb the PNS surface. This is also the cause of the second signal component of the signals presented in this thesis.  

Unlike the high-frequency emission, low-frequency emission \citep{kuroda_16} 
is not present in all models. It is unik to the models where SASI activity develops.
We found that the whole simulation volume contributes to this emission component. The SASI imprints the flow in the post-shock layer with a
large-scale asymetric pattern, which directly leads to GW emission. The viloent sloshing, and spiral motions of the post-shock layer
creates choherent modulations of the accretion flow onto the PNS. When matter carried by the SASI slams into the surface of the PNS
it forces mass motions in the surface of the star that propegates deep into the PNS. These forced mass motions leads to emission of GWs.
Since it is the SASI that creates these flow patterns, the frequency is naturally set by the typical time scale of the SASI ossilations.

When studying the effect of moderate rotation, we found that there is no siginficant effect on the over all structure of the
GW signal. Unlike what is found for rapdilty rotating models \citep{rampp_98,shibata_05,ott_05,kuroda_14,takiwaki_16}, 
we do not find any enhancment of the GW signal in the slower rotating models we study here. Rotation seems to impact the 
strength of the GW signal mainly through its interplay with the spiral SASI mode. It might be that rotation aids
the growth of SASI activity \citep{yamasaki_08,blondin_17}, however it is not inherently clear that this should be the case
for all cases \citep{kazeroni_17}.

Because rotation has a stabilising effect on PNS convection, we find a reduction of the high-frequency emission generated
by PNS convection. In the rotating model (m15r) where SASI activity dose not develop there is a strong reduction of the high-frequency
emission. The weak GW signal from this model confirms the fact that post-shock convection, which becomes the dominate source of GW emission once
PNS convection and SASI activity has been eliminated, is a weak source.


