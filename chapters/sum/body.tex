\chapter{Summary and discussion}
\section{Summary}
In this thesis, we have studied the GW signals from the accretion phase and the early
explosion phase of core-collapse supernovae based on seven 3D
multi-group neutrino hydrodynamics simulations. The seven models are based on four progenitors
with ZAMS masses of $11.2 M_\odot$, $20 M_\odot$, $27 M_\odot$, and $15 M_\odot$.
Two models based on the $15 M_\odot$ progenitor included moderate initial rotation, while the
rest of the models were non-rotating.

In general, we find that the signals consist of two distinct emission components. A high-frequency component that is present in the signals from all the models. This emission is generated
by buoyancy effects in the PNS surface. Instigated either by the overshooting of convective plumes
from PNS convection or by downflows from the post-shock layer. The typical frequency of this component is
given by the Brunt-V\"{a}is\"{a}l\"{a}-frequency which traces the properties of the PNS, such as the mass, radius, 
and surface temperature of the PNS, see \cite{mueller_13}.
In the models were the SASI develops we find strong emission below 250 Hz. This emission stems from the
global modulation of the accretion flow by the SASI. 
Mass motions in the post-shock region, the PNS surface, and the inner regions
of the PNS all contribute to this signal component. The strong downflows generated by SASI activity effectively perturbs the PNS surface and these
perturbations propagate deep into the PNS.

We found, contrary to the results from 2D \citep{marek_08,murphy_09,mueller_13}, that most of the high-frequency emission is generated by PNS convection and that downflows from the post-shock layer only play a minor role. The reason why excitation of g-modes by downflows is less predominant in 3D is twofold. Firstly, in 2D energy cascades towards larger and larger scales, unlike
in 3D where energy cascades towards smaller and smaller scales. This means that 2D simulations more easily develop large-scale, and
high-velocity, downflows in the post-shock layer that more efficiently perturb the PNS surface. Since energy is artificially concentrated at large scales, the typical timescales of variations in the mass motions around the PNS surface are shorter in 2D than in 3D. Consequently, the spectrum of the forcing overlaps better with the natural frequency
of surface g-modes in 2D.
In the models dominated by the spiral SASI mode we find particular strong high-frequency emission. These models
also emit high-frequency GWs in a wider frequency range. The spiral SASI mode leads to coherent and strong downflows that are able to
strongly perturb the PNS surface. This is also the cause of the low-frequency component that we found in the GW signals.  

In chapter~\ref{ch:paper2} we studied the effect of moderate progenitor rotation and found that
there is no significant change in the frequency structure of the GW signal when rotation is included, compared to the signal from non-rotating models. We do, however, find that the high-frequency emission instigates by PNS convection in weaker in the rotating models.
Rotation acts as a stabiliser in the PNS convection layer and leads to a decrease in the amount of energy dissipated in the overshooting region PNS.
While the fastest rotating model emits the strongest GW signal, we are not able to find a clear connection between rotation rate and
the strength of the GW amplitudes. This is in contrast to what is found in rapidly rotating models rapidly rotating models 
\citep{rampp_98,shibata_05,ott_05,kuroda_14,takiwaki_16}.

We calculated the possibility of detecting the GW signals presented here. In general, we conclude that with current detectors it will be very difficult to detect the signals. The typical amplitudes are
on the order of a few centimeter, which is very weak compared to other sources (such as black hole binary systems).
With next generation instruments, like the Einstein telescope, detection should be possible for an event within the 
Milky way. Not only will it be possible to detect the events, but it will be possible to
differentiate between models without and with low-frequency emission.

\section{Uncertainties}
The uncertainties of the gravitational wave signal are ultimately connected to the certainties
of the underlying supernovae models. Recently we have seen the emergence of the first successful explosions 
in 3D simulations \citep{melson_15a,melson_15b,lentz_15,suma_models}. 
These first explosion models prove that the delayed neutrino-driven explosion mechanism can produce explosions in full 3D. 
However, we are not yet at the point where 3D simulations robustly produce explosions. 
There are several sources of uncertainty in the current core-collapse simulations and it might be useful to divide the uncertainties into two categories. 

The first category of uncertainties are uncertainties related to the input physics. 
It has been shown that asymmetries in the burning layers of the stellar progenitor can influence the dynamics of the core-collapse. 
The simulations that the gravitational wave signal presented in this thesis are based on all start from spherical symmetric progenitors
\citep{burrows_96,fryer_04,arnett_11,couch_13,mueller_15a}. 
Another example in uncertain the input physics is the higher density equation of state. 
How matter behaves at such high densities and how neutrinos interact with the hot, and dense 
stellar matter a subject where much is still uncertain \citep{fischer_14,lattimer_16}.

The second category contains the technical problems with the simulations. That is to say the uncertainties connected to the current implementation of the physics included in the codes. 
While the ray-by-ray+ approximation implemented in \textsc{Prometheus-Vertex} \citep{rampp_02} is state of the art it is still only an approximate
solution of the full radiation problem. The one-dimensional approach of the ray-by-ray+ method fails to fully account for
transverse fluxes and lateral radiation transport. \cite{skinner_16} performed 2D simulations using both the ray-by-ray+ approximation and a multi-dimensional transport scheme. They found significantly different results from the two setups.
Another technical problem is the issue of grid resolution. It is not clear that the current resolution is sufficient to resolve turbulence in the post-shock layer. It is still debate of how resolution affects the simulation outcome, and
how high the spatial resolution needs to be in order to accurately simulate the core-collapse of a massive star. For a detailed discussion
of the subject see Chapter~7 of \cite{melson_phd} and references therein. 

In their closing remarks \cite{skinner_16} rather aptly writes:
\begin{displayquote}
\textit{In fact, there is a rather long list of numerical
challenges and code verification issues yet to be met
collectively by the world’s supernova modelers. The results
of different groups are still too far apart to lend ultimate
credibility to any one of them.} 
\end{displayquote}
What ultimately proves to be the solution to the explosion problem in 3D is not clear. Maybe a
missing physical ingredient turns out to be the solution, or maybe there is a need for
more accurate numerical solutions. A likely scenario is that the solution is
a combination of the two. 

On the other hand, based on current 3D simulations, we could draw the conclusion that the neutrino-driven explosion mechanism only needs a small push in order to be successful. Any of the aforementioned effects might provide this push. 
In model s20s the interaction rates of the neutrinos were slightly modified \citep{melson_15b}, this resulted in a successful explosion. 
Similarly, in model m15fr a successful explosion was with achieved with the help of rotation \citep{suma_models}. 
This indicates that we should not expect the drastic change in the dynamics of the core collapse scenario, 
but rather small changes of the details in the simulations. For example the strength of the SASI spiral mode. 
Consequently, we should expect that the gravitational wave signals presented here, at least qualitatively, 
captures the essence of the gravitational radiation emitted by core collapse supernovae. \cite{mueller_13} found systematic
shifts in the typical emission frequencies when comparing full GR simulations to pseudo-Newtonian, and Newtonian simulations.


\section{Outlook}
Future work in this field will, naturally, consist of continued prediction and study of the GW signals produced by core collapse simulations. 
As the simulations improve so will the predictions for the gravitational wave signals. 
We have in this thesis, rather crudely, estimated the possibilities for detecting the signals we studied. 
These estimates do not give us grounds for optimism. In current detectors, it would be difficult to detect even galactic events. 
However, it might be possible to develop more sophisticated algorithms and methods to enhance detection capabilities of the detectors. 
The bright electromagnetic signal associated with core-collapse supernovae gives us an advantage since it will be possible to pinpoint the location of the signal in the sky and with fairly high accuracy determine the time during which we should search for a signal in the data. In addition to the electromagnetic signal, 
for a closed event neutrinos will provide us with even tighter constraints on the time window. 
While the signal from core collapse supernova at first glance seems rather stochastic there is a significant degree of structure within the signal. This means that we can search for a particular type of signal in the detector data. 

In terms of the understanding of the underlying processes responsible for GW emission in the simulations, the exact nature
of the interaction between the SASI and the PNS is not yet fully understood. A more thorough, and rigorous study of
how downflows affect the PNS surface is necessary to better understand both the low-frequency and the high-frequency emission.
It will also be important to determine how the PNS convection layer is influenced by accretion and how this affects
the high-frequency emission.
