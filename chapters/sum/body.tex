%%% Local Variables:
%%% mode: latex
%%% TeX-master: "../../doktorarbeit"
%%% End:
\chapter{Summary and discussion}
\paragraph{To summarise,}
in this thesis we have studied the GW signals predicted
by seven three-dimensional core-collapse supernovae simulations, with state of the art 
neutrino transport. The main focus has been to identify the fingerprints in the signal
from hydrodynamic instabilities operating behind the stalled shock front, in the time between the
recoil of the inner core and the onset of shock revival. However, since two of the simulations
result in successful explosions, we have also studied the evolution of the GW emission
for the first few hundred milliseconds after shock revival. Two of the seven models, one of the two
yielding a successful explosion, were based on rotating progenitors. The models cover both the
regime where the post-shock region is dominated by neutrino driven convection, and the regime where
the so called standing accretion shock instability dominates the post-shock flow.

The GW emission from the models is dominated by two distinct components. In all models
GWs are emitted at frequencies greater than 250 Hz, what we in this thesis refer to as 
high-frequency emission. The peak frequency, the frequency at which the maximum amount of radiation 
is emitted, of this emission component increases as the simulations progress forward in time. 
This kind of emission had previously been seen in 2D simulations \citep{marek_08,murphy_09,mueller_13}
and have been connected to excitation of g-mode oscillations in the convectively stable outer layers of the PNS.
In 2D, the main source of these g-modes was identified to be downflows from the post-shock layer impinging onto the PNS surface.
The overshooting of convective plumes from the convectively unstable layer within the PNS was found to be a second, but sub-dominant, exaction
mechanism. PNS convection was only after the onset of shock revival, because of a significant drop in the accretion rate, found to 
be dominating.  The typical angular frequency of such g-modes roughly given by the Brunt-V\"{a}is\"{a}l\"{a}-frequency
the convectively stable outer layers of the PNS. \citet{mueller_13} investigated the dependence
of the Brunt-V\"{a}is\"{a}l\"{a}-frequency on the mass, radius, and surface temperature and explain the secular
increase of the frequency during the contraction of the PNS.

We found, contrary to the results from 2D \citep{marek_08,murphy_09,mueller_13}, that most of the high-frequency emission is 
generated by PNS convection and that downflows from outside of the PNS only player a minor role. The reason why excitation 
of g-modes by downflows is less predominant in 3D is twofold. Firstly, in 2D energy cascades towards larger and larger scales, unlike
in 3D where energy cascades towards smaller and smaller scales. This means that 2D simulations more easily develop large-scale, and
high-velocity downflows in the post-shock layer, that more efficiently perturb the PNS surface. The second reason is connected to the first, since
energy is artificially concentrated at large and intermitten scales, the typical timescales of variations in the mass motions around 
the PNS surface are shorter in 2D than in 3D. Consequently, the spectrum of the forcing overlaps better with the natural frequency
of surface g-modes in 2D, better than in 3D.
In the models where a strong spiral SASI mode dominates the post-shock flow we find particular strong high frequency emission. These models
also emits high-frequency GWs in a wider frequency range. The spiral SASI mode leads to coherent and strong downflows that are able to
strongly perturb the PNS surface. This is also the cause of the second signal component of the signals presented in this thesis.  

Unlike the high-frequency emission, low-frequency emission \citep{kuroda_16} 
is not present in all models. It is unik to the models where SASI activity develops.
We found that the whole simulation volume contributes to this emission component. The SASI imprints the flow in the post-shock layer with a
large-scale asymmetric pattern, which directly leads to GW emission. The viloent sloshing, and spiral motions of the post-shock layer
creates coherent modulations of the accretion flow onto the PNS. When matter carried by the SASI slams into the surface of the PNS
it forces mass motions in the surface of the star that propagates deep into the PNS. These forced mass motions leads to emission of GWs.
Since it is the SASI that creates these flow patterns, the frequency is naturally set by the typical time scale of the SASI oscillations.

When studying the effect of moderate rotation, we found that there is no significant effect on the over all structure of the
GW signal. Unlike what is found for rapidly rotating models \citep{rampp_98,shibata_05,ott_05,kuroda_14,takiwaki_16}, 
we do not find any enhancement of the GW signal in the slower rotating models we study here. Rotation seems to impact the 
strength of the GW signal mainly through its interplay with the spiral SASI mode. It might be that rotation aids
the growth of SASI activity \citep{yamasaki_08,blondin_17}, however it is not inherently clear that this should be the case
for all cases \citep{kazeroni_17}.

Because rotation has a stabilising effect on PNS convection, we find a reduction of the high-frequency emission generated
by PNS convection. In the rotating model (m15r) where SASI activity dose not develop there is a strong reduction of the high-frequency
emission. The weak GW signal from this model confirms the fact that post-shock convection, which becomes the dominate source of GW emission once
PNS convection and SASI activity has been eliminated, is a weak source of GW emission.

We calculated the possibility of detecting the GW signals presented here. In general, we conclude that 
with current detectors it will be very difficult to detect the signals. The typical amplitudes are
on the order of a few centimetre, which is very weak compared to other sources (such as black hole binary systems).
With next generation instruments, like the Einstein telescope, detection should be possible for an event within the 
Milky way. Not only will it be possible to detect the events, but we estimate that it will be possible to
differentiate between models without and with low-frequency emission.

\paragraph{The uncertainties}of the gravitational wave signal is ultimately connected to the certainties
of the underlying supernovae models. Recently we have seen the emergence of the first successful explosions 
in 3D simulations \citep{melson_15a,melson_15b,lentz_15,suma_models}. 
These first explosion models proves that the delayed neutrino driven explosion mechanism can produce explosions in full 3D. 
However, we are not yet at the point where 3D simulations robustly produce explosions. 
There are several sources of uncertainty in the current core-collapse simulations and 
it might be useful to divide the uncertainties into two categories. 

The first category of uncertainties are uncertainties related to the input physics. 
It has been shown that asymmetries in the burning layers of the stellar progenitor can influence the dynamics of the core-collapse. 
The simulations that the gravitational wave signal presented in this thesis are based on all start from spherical symmetric progenitors
\citep{burrows_96,fryer_04,arnett_11,couch_13,mueller_15a}. 
Another example in uncertain the input physics is the higher density equation of state. 
How matter behaves at such high densities and how neutrinos interact with the hot, and dense 
stellar matter a subject where much is still uncertain \citep{fischer_14,lattimer_16}.

The second category contains the technical problems with the simulations, 
that is to say the uncertainties connected to the current implementation of the physics included in the codes. 
While the ray-by-ray+ approximation implemented in \textsc{Prometheus-Vertex} \citep{rampp_02} is state of the art it is still only an approximate
solution of the full radiation problem. The one-dimensional approach of the ray-by-ray+ method fails to fully account for
transverse fluxes and lateral radiation transport. \cite{skinner_16} performed 2D simulations using both the ray-by-ray+ approximation,
and a multi-dimensional transport shceem. They found significantly quantitative and qualtivity different results from the two setups.
Another technical problem is the issue of grid resolution. It is not clear that the current resolution is 
sufficient to resolve turbulence in the post-shock layer. It is still debate of how resolution effects the simulation outcome, and
how high the spatial resolution needs to be in order to accurately simulate the core-collapse of a massive star. For a detailed discussion
about the subject see Chapter~7 of \cite{melson_phd} and references therein. 

In their closing remarks \cite{skinner_16} rather aptly writes:
\begin{displayquote}
\textit{In fact, there is a rather long list of numerical
challenges and code verification issues yet to be met
collectively by the world’s supernova modelers. The results
of different groups are still too far apart to lend ultimate
credibility to any one of them.} 
\end{displayquote}
What ultimately proves to be the solution to the explosion problem in 3D is not clear. Maybe a
missing physical ingredient turns out to be the solution, or maybe there is a need for
more accurate numerical solutions. A likely scenario is that the solution is
a combination of the two. 

On the other hand, based on current 3D simulations, we could draw the conclusion that the Neutrino driven explosion 
mechanism only needs a small push in order to be successful. Any of the aforementioned effects might provide this push. 
In model s20s the interaction traits of neutrinos was slightly modified \citep{melson_15b}, this resulted in a successful explosion. 
Similarly in model m15fr a successful explosion was with achieved with the help of rotation \citep{suma_models}. 
This indicates that we not should not expect the drastic change in the dynamics of the core collapse scenario, 
but rather small changes of the details of the simulations. For example the strength off the SASI spiral mode. 
Consequently, we should expect that the gravitational wave signals presented here, at least qualitatively, 
captures the essence of the gravitational radiation emitted by core collapse supernovae. 

\paragraph{Future work} will consist a continued prediction and study of gravitational waves signals produced by core collapse simulations. 
As the simulations improve so will the predictions for the gravitational wave signals. 
We have in this thesis, rather crudely, estimated the possibilities for detecting the signals we studied. 
These estimates yields rather negative results, in current detectors it would be difficult to detect even galactic events. 
However, it might be possible to develop more sophisticated algorithms and methods to enhance detection capabilities of the detectors. 
The bright electromagnetic signal associated with core collapse supernovae gives us an advantage since it will be possible 
to pinpoint the location of the signal on the sky and with fairly high accuracy determine the time in which we 
should search for our signal in the data. In addition to the electromagnetic signal, 
for a closed event neutrinos will provide us with even tighter constraints on the time window. 
While the signal from core collapse supernova at first glance seems rather stochastic there is 
a significant degree of structure within the signal. This means that we can 
search for a particular type of signal in the detector data. 

In terms of the understanding of the underlying processes responsible for GW emission in the simulations, the exact nature
of the interaction between the SASI and the PNS is not yet fully understood. A more thorough, and rigorous study of
how downflows affect the PNS surface is necessary to better understand both the low-frequency, and the high-frequency emission.
It will also be important to determine how the PNS convection layer is influenced by accretion and how this affects
the high-frequency emission.