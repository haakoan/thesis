\chapter{Conclusions}
\section{Summary and discussion}
In this thesis, we have studied the GW signals from the accretion phase and the early
explosion phase of core-collapse supernovae based on seven 3D
multi-group neutrino hydrodynamics simulations. The seven models are based on four progenitors
with ZAMS masses of $11.2 M_\odot$, $20 M_\odot$, $27 M_\odot$, and $15 M_\odot$, respectively.
Two models, based on the $15 M_\odot$ progenitor, included moderate initial rotation, while the
rest of the models are non-rotating.

Broadly speaking, the GW signal from 3D simulations can be described as stochastic signals with 
amplitudes of few a centimetres, that consists of two distinct signal components.
The first component, which is present in all of the signals, consists of emission at frequencies
above 250 Hz (high-frequency emission). The typical frequency of this emission increases almost 
linearly as a function of time. The second component is only seen in the models where
SASI activity develops. During phases of strong SASI activity 
emission in the range between $100 \, \mathrm{Hz}$ and $200 \, \mathrm{Hz}$ (low-frequency) 
emerges as a characteristic feature of the GW signal.

Low-frequency emission, during the accretion phase, 
originates from the global modulation of the accretion flow by the SASI.
In the layer between the forming neutron star and the shock front, the SASI creates a coherent
asymmetric mass distribution with a large quadrupole moment. The density perturbations induced
by the SASI have the same temporal structure
as the shock oscillations. This leads to the emission of GWs from the post-shock layer with a typical frequency which is proportional to the SASI frequency. 
Furthermore, strong low-frequency GWs emission is instigated when the anisotropic modulation of the accretion flow is felt by the PNS when matter is accreted onto the PNS. 
In the surface layer of the PNS, mass motions are forced by the strong downflows that result from SASI activity. The perturbation of the PNS surface leads to
GW emission in a similar, but somewhat broader, frequency range as the emission from the
post-shock layer. GW emission continues as the accreted matter is advected deeper into the 
PNS, since the density, and entropy perturbations are not
completely erased by neutrino cooling. Density fluctuations around the one percent level
are maintained even as the perturbations reaches the inner regions of the PNS. When propagating down into the interior of the PNS, these perturbations roughly maintain the frequency spectrum set by the SASI.

These conclusions are in line with the very recent study of \cite{kuroda_16}, where it is suggested that low-frequency emission is a fingerprint of SASI activity. However, we find that the exploding model s20s
emits strong low-frequency GWs after the onset of shock expansion. During this time
there is clearly no SASI activity. We trace this emission to the interior of the PNS and
attribute it to the fact that the flow in the PNS convective layer transitions from
a dipole dominated pattern to one where the quadrupole mode dominates.
We speculate that this could result from changes in the accretion flow onto the PNS,
or shifts in the entropy and electron fraction of the PNS.

By analysing two 2D simulations of the 27 $M_\odot$ progenitor, we find that low-frequency 
emission also exist in the 2D models. However, it is completely overshadowed by the high-frequency emission and
has, therefore, not been emphasised in recent 2D studies \citep{marek_08,murphy_09,mueller_13}.

The typical frequency of the high-frequency component closely traces
the Brunt-V\"{a}is\"{a}l\"{a}-frequency of the outer layer of PNS,
the roughly isothermal atmosphere layer between the PNS convection zone 
and the post-shock layer. In 2D this emission has been found to
be the result downflows from the post-shock layer impinging onto the PNS surface \citep{marek_08,murphy_09,mueller_13}.
However, in the 3D models studied in this work the high-frequency signal mostly originates from aspherical mass motions
in the overshooting region of PNS. Convective plumes from the PNS convection zone overshoot into the convectively stable layer above. When these plumes are decelerated and overturned high-frequency GWs are emitted. 
Downflows from the post-shock layer contribute only weakly to the total high-frequency signal. 
We attribute the difference between 2D and 3D mainly to the inverse cascade of turbulent energy in 2D. 
In 2D energy cascades towards large and large scales which leads to the development of
large downflows with impact larger velocities. These large flow structures effectively excite resonant g-mode oscillations in the PNS surface that give rise to GW emission.
In 3D, braking of downflows by the forward turbulent cascade results in the fragmentation of large eddies into smaller structures
and this suppress surface g-mode excitation. Furthermore, when large and medium sized eddies are broken up into smaller
eddies this changes the spectrum of the turbulent mass motions in post-shock layer.
This means that the spectrum of the of the forcing does not extend to high frequencies in 3D.
Excitation of surface g-modes at their eigenfrequency, therefore, becomes
ineffective.

In chapter~\ref{ch:paper2} we studied the effect of moderate progenitor rotation, we studied three simulations 
of a 15 $M_\odot$ progenitor.
Including moderate progenitor rotation does not significantly change the frequency structure or overall amplitudes of the GW signal, 
compared the signals from non-rotating models. We do, however, find that the high-frequency emission instigated by PNS convection is weaker in the two rotating models. Rotation acts as a stabiliser of the PNS convection layer and leads to a decrease in the amount of energy dissipated in the overshooting region PNS. 

In regards to the signal strength as a function of the progenitor rotation rate, we do not find
a clear trend of increasing signal strength with increasing initial rotation rate.
Of the three simulations based on the 15 $M_\odot$ progenitor, the fastest rotating model emits the strongest GW signal. On the other hand,
the non-rotating version emits a stronger signal than the slowest rotating model.
The reason why the fastest rotating model emits the strongest GW signal is because it develops the strongest SASI oscillations.
It is possible that rotation aids the development of the spiral SASI mode, but it remains unclear
whether increasing the rotation rate always leads to stronger SASI activity.

The fastest rotating model emits the strongest GW signal because it develops the strongest spiral SASI mode. Based on the models presented here we can not conclude
that the strength of the signal increase with increasing progenitor rotation rate. The conclusion should instead be
that the strong SASI activity leads to strong GW emission. However, it remains unclear
whether faster rotation leads to stronger SASI activity. In the linear regime, the growth rate of SASI modes has found to
increase as a function of rotation \citep{yamasaki_08,blondin_17}. However, in the non-linear regime, \cite{kazeroni_17} does not find a 
monotonic connection between increasing rotation rate and the saturation amplitude SASI modes. The results of \cite{kazeroni_17} indicates that
SASI activity may decrease with increasing rotation rate, at least at low to moderate rotation rates. 

We calculated the possibility of detecting the GW signals presented here. In general, we conclude that with current detectors it will be very difficult to detect the signals. The typical amplitudes are
on the order of a few centimeters which are very small compared to that of other sources (such as black hole binary systems).
With next generation instruments, like the Einstein telescope, detection should be possible for an event within the 
Milky way. Not only will it be possible to detect the events, but it will be possible to
differentiate between models without and with low-frequency emission.

\section{Uncertainties}
The uncertainties of the gravitational wave signal are ultimately connected to the uncertainties
of the underlying supernovae models. Recently we have seen the emergence of the first successful explosions 
in 3D simulations \citep{melson_15a,melson_15b,lentz_15,suma_models}. 
These first explosion models prove that the delayed neutrino-driven explosion mechanism can produce explosions in full 3D. 
However, we are not yet at the point where 3D simulations robustly produce explosions. 
There are several sources of uncertainty in the current core-collapse simulations and it might be useful to divide these uncertainties into two categories. 

The first category of uncertainties is related to the input physics. 
It has been shown that asymmetries in the burning layers of the stellar progenitor can influence the dynamics of the core-collapse. 
The simulations, on which the gravitational wave signal presented in this thesis are based on, all start from spherical symmetric progenitors
\citep{burrows_96,fryer_04,arnett_11,couch_13,mueller_15a}. 
Another example of uncertain input physics is the high-density equation of state. 
How matter behaves at super-nuclear densities and how neutrinos interact with the hot, and dense 
stellar matter is uncertain \citep{fischer_14,lattimer_16}.

The second category concerns technical problems with the simulations. 
That is to say, the uncertainties connected to the current implementation of the physics included in the codes. 
While the ray-by-ray+ approximation implemented in \textsc{Prometheus-Vertex} \citep{rampp_02} is state of the art, it is still only an approximate
solution of the full radiation problem. The one-dimensional approach of the ray-by-ray+ method fails to fully account for
transverse fluxes and lateral radiation transport. \cite{skinner_16} performed 2D simulations using both the ray-by-ray+ approximation and a multi-dimensional transport scheme. They found significantly different results for the two setups.
Another technical problem is the issue of grid resolution. Currently, even the best resolved simulations can not hope to sufficiently resolve turbulence in the post-shock layer.
It is still debate of how resolution affects the simulation outcome, and
which the spatial resolution is needed to accurately simulate the core-collapse of a massive star. 
For a detailed discussion of the subject see Chapter~7 of \cite{melson_phd} and references therein. 

In their closing remarks \cite{skinner_16} rather aptly write:
\begin{displayquote}
\textit{In fact, there is a rather long list of numerical
challenges and code verification issues yet to be met
collectively by the world’s supernova modelers. The results
of different groups are still too far apart to lend ultimate
credibility to any one of them.} 
\end{displayquote}
What ultimately proves to be the solution to the explosion problem in 3D is not clear. Maybe a
missing physical ingredient turns out to be the solution, or maybe there is a need for
more accurate numerical solutions. A likely scenario is that the solution is
a combination of the two. 

Recent 3D models have shown that the neutrino-driven explosion mechanism can indeed produce successful explosion. 
\citep{melson_15b} presented a simulation of a non-rotating 20 $M_\odot$ progenitor. In this simulation the interaction rates of the neutrinos were slightly modified, resulting in a successful explosion. 
Similarly, a successful explosion was achieved with the help of rotation by \citep{suma_models}. 
This indicates that we should not expect a drastic change in the dynamics of the core-collapse scenario, 
but rather small changes of the details in the simulations. 
Consequently, we should expect that the GW signals presented here, at least qualitatively, 
capture the essence of the gravitational radiation emitted by core collapse supernovae. Here we should mention the study of \cite{mueller_13}, in which the authors found systematic shifts in the typical emission frequencies when comparing full GR simulations to pseudo-Newtonian, and Newtonian simulations.
While the pseudo-Newtonian approximation of gravity that the simulations utilise is sufficient to capture the dynamics of the core-collapse, full
GR simulations will be needed to determine the exact numerical value of the GW frequencies, and amplitudes.

\section{Outlook}
Future work in this field will, naturally, consist of improved predictions and studies of the GW signals produced by core-collapse simulations. 
As the simulations improve so will the predictions for the gravitational wave signals. 
We have in this thesis, rather crudely, estimated the possibilities for detecting the signals we studied. 
These estimates do not give us grounds for optimism. In current detectors, it would be difficult to detect even galactic events. 
However, it might be possible to develop more sophisticated algorithms and methods to enhance detection capabilities of the detectors. 
The bright electromagnetic signal associated with core-collapse supernovae gives us an advantage since it will be possible to pinpoint the location of the signal in the sky and with fairly high accuracy determine the time during which we should search for a signal in the data. In addition to the electromagnetic signal, 
for a closed event neutrinos will provide us with even tighter constraints on the time window. 
While the signal from core collapse supernova at first glance seems rather stochastic there is a significant degree of structure within the signal. This means that we can search for a particular type of signal in the detector data. 

In terms of the understanding of the underlying processes responsible for GW emission in the simulations, the exact nature
of the interaction between the SASI and the PNS is not yet fully understood. A more thorough, and rigorous study of
how downflows affect the PNS surface is necessary to better understand both the low-frequency and the high-frequency emission.
It will also be important to determine how the PNS convection layer is influenced by accretion and how this affects
the high-frequency emission.

In this thesis, we have taken a first step towards studying the GW signals from slow/moderately rotating supernovae models in 3D. In the future, it will be necessary do perform more systematic studies of the effects of moderate progenitor rotation in order to determine exactly how
the rotation rate of the progenitor affects the spiral SASI mode, and the GW signal.
