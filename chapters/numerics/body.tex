
%%% Local Variables:
%%% mode: latex
%%% TeX-master: "../../doktorarbeit"
%%% End:
\chapter{Numerical simulations of core-collapse supernovae}
The GW signals presented in this thesis is based on 
supernova simulations that were performed with the code \textsc{Prometheus-Vertex}.
The simulations were carried out by members of the Garching-group 
(\cite{hanke_phd,melson_phd,suma_models}).
In this chapter we described the code that was used, we will
describe the details of individual models in later chapters. 
\textsc{Prometheus-Vertex} solves the combined problem of hydrodynamics and neutrino radiation transport.
The code has been specifically developed for the core-collapse problem,
the first version was developed in 2002 by \cite{rampp_02} 

\section{\textsc{Prometheus}}
The hydrodynamics of the stellar collapse is solved with a version of the
well-established \textsc{Prometheus} code \citep{mueller_91,fryxell_91}. It solves the non-relativistic 
equations of hydrodynamics for the inviscid flow of an ideal fluid in spherical coordinates $(r,\theta,\phi)$.
\subsection{Hydrodynamics}
In spherical coordinates the Euler equations of hydrodynamics are 
\begin{alignat}{1}
&\partial_t \rho + \frac{1}{r^2} \partial_r (r^2 \rho v_r) + \frac{1}{r \sin{\theta}} \bigg[ \partial_{\theta} (\rho \sin{\theta} v_{\theta}) + \partial_{\phi} (\rho v_{\phi}) \bigg] = 0, \label{eqN:cont} \\
\stepcounter{equation}
&\partial_t (\rho v_r) + \frac{1}{r \sin{\theta}} \bigg[ \partial_{\theta} (\rho \sin{\theta} v_{\theta} v_r)
+\partial_{\phi} (\rho v_{\phi} v_r) \bigg]  + \nonumber \\ 
&\frac{1}{r^2} \partial_r (r^2 \rho v_r^2)  - \rho \frac{v_{\theta}^2 + v_{\phi}^2}{r} + \partial_r p 
= - \rho \partial_r \Phi + Q_{M_r}, \tag{\theequation{}a} \\
%
&\partial_t (\rho v_{\theta}) + \frac{1}{r^2} \partial_r (r^2 \rho v_r v_{\theta}) + \frac{1}{r \sin{\theta}}\bigg[ \partial_{\theta} (\rho \sin{\theta} v_{\theta}^2) +
\partial_{\phi} (\rho v_{\phi} v_{\theta}) \bigg] \nonumber \\
&+\rho \frac{v_{\theta}v_r - v_{\phi}^2/\tan{\theta}}{r} + \frac{1}{r}\partial_{\theta} p = - \frac{\rho}{r} \partial_{\theta} \Phi + Q_{M_{\theta}}, \tag{\theequation{}b} \\
%
&\partial_t (\rho v_{\phi})  
+\frac{1}{r \sin{\theta}} \bigg[ \partial_{\theta} (\rho \sin{\theta} v_{\theta} v_{\phi}) +
\partial_{\phi} (\rho v_{\phi}^2)  + \partial_{\phi} p  \bigg] \nonumber \\
&+ \frac{1}{r^2} \partial_r (r^2 \rho v_r v_{\phi}) +
\rho \frac{v_{\theta}v_{\phi}/\tan{\theta} + v_{\phi}v_r}{r}
 = -\frac{1}{r \sin{\theta}} \partial_{\phi} \Phi + Q_{M_{\phi}}, \tag{\theequation{}c} \\
%
\stepcounter{equation}
&\partial_t e + \frac{1}{r^2} \partial_r (r^2 v_r (e + p))
+\frac{1}{r \sin{\theta}} \bigg[ \partial_{\theta} ( \sin{\theta} v_{\theta} (e + p))  +
\partial_{\phi} (v_{\phi} (e + p)) \bigg] \nonumber \\
&= -\rho v_r \partial_r \Phi - \rho\frac{v_{\theta}}{r}\partial_{\theta}\Phi - \frac{v_{\phi}}{r \sin{\theta}} \partial_{\phi} \Phi
 + Q_E + v_r Q_{M_r} + v_{\theta} Q_{M_{\theta}} + v_{\phi} Q_{M_{\phi}}.
\end{alignat}
Here $p$ is pressure, $\rho$ is density, $e$ is the specific total energy,  $\Phi$ is the gravitational potential and
$v_r$, $v_{\theta}$, and $v_{\phi}$ are the velocity components in the spherical
coordinate basis. The source terms $Q_{M_{r}}$, $Q_{M_{\theta}}$, and $Q_{M_{\phi}}$ represent
momentum transfer by neutrinos, in the radial, polar, and azimuthal direction, respectively.
Energy transport by neutrinos is represented by the source term $Q_{M_{E}}$. These source terms are calculated
by the neutrino-transport module \textsc{Vertex}, which we will come back to them later in the chapter.
The above equations have to be closed by an EoS, which generally will depend on 
density, internal energy, and the chemical composition of the stellar matter.
This means that one has to track of two additional quantities, the mass fractions
of different nuclear species, denoted by $X_i$, and the electron fraction $Y_e$. We, therefore,
have to solve two additional conservation equations:
\begin{alignat}{1}
\partial_t (\rho X_i) + \frac{1}{r^2} \partial_r (r^2 \rho v_r X_i) + \frac{1}{r \sin{\theta}} \bigg[ \partial_{\theta} (\rho \sin{\theta} v_{\theta} X_i) + \partial_{\phi} (\rho v_{\phi} X_i) \bigg] &= \varsigma_i, \label{eqN:contxi} \\
\partial_t (\rho Y_e) + \frac{1}{r^2} \partial_r (r^2 \rho v_r Y_e) + \frac{1}{r \sin{\theta}} \bigg[ \partial_{\theta} (\rho \sin{\theta} v_{\theta} Y_e) + \partial_{\phi} (\rho v_{\phi} Y_e) \bigg] &= Q_{Y_e}. \label{eqN:contye}
\end{alignat}
The two terms $\varsigma_i$ and $Q_{Y_e}$ represents the change of composition of species $i$ due nuclear reactions
and the change in electron fraction caused by emission and absorption of electron and anti-electron neutrinos,
respectively. If the fluid reaches nuclear statistical equilibrium (NSE), then the chemical 
composition is fully determined by the equation of state, through the electron fraction, density, and
temperature.

\textsc{Prometheus} solves the system of equations described in the paragraph above
by means of a dimensionally-split implementation of the piecewise parabolic method of \cite{colella_84}.
The scheme is time-explicit and it is accurate to third-order in space and second-order in time.
The Riemann solver exactly solves 1D Riemann problems in so-called sweeps
that have been obtained from the full 3D equations by Strange-splitting \citep{strang_68}.
When strong shocks are encountered the solver switches to the ``HLLE'' solver \citep{einfeldt_88}.
This is done in order to avoid the so-called ``even-odd decoupling'', which occurs
when the shocks are aligned with one of the axes of the simulation grid \citep{quirk_94,kifonidis_03},
which creates artificial oscillations. \textsc{Prometheus} employs the 
consistent multi-fluid advection method of \cite{plewa_99} to ensure that the advection all the nuclear species
is calculated accurately.

\subsection{Equation of state}
The models that this work is based on uses two different prescriptions
for the EoS. A ``high-density'' EoS is used for the inner
hot region of the core, while a ``low-density'' EoS is used for the low-density regions of the simulation volume. The two are separated by a density threshold,
after core bounce this threshold is set to $\rho_T = 10^{11}$ g/cm$^3$.
In the high-density regime the tabular EoS of 
\cite{lattimer_91} with a nuclear incompressibility of 220 MeV is used. 
Below the threshold, in the low-density regime, an EoS that describes the nuclei as
classical Boltzmann gases. Electrons and positrons are described as Fermi gases with arbitrary degeneracy levels. The EoS also include the effect of photons \citep{janka_99}.

\subsection{Self gravity - the effective potential}
Self-gravity is treated using the monopole                      
approximation and the effects of general relativity are accounted for                                               
in an approximate fashion by means of a pseudo-relativistic                                                          
effective potential (case~A of \cite{marek_06}). 
The effective potential includes general relativistic effects and
takes the properties of the medium into account, such as the pressure and energy
of the fluid elements. In the case of a monopole potential, the terms involving
non-radial derivatives of the potential in the Euler equations vanish.

\section{\textsc{Vertex}}
The \textsc{Vertex} code calculates the source terms on the right-hand side of the
Euler equations by treating the neutrinos as a radiation field. 
The problem we need to solve is to find the phase-space distribution function $f(\mathbf{r},\mathbf{q},t)$ of the neutrinos.
Essentially we want to find the number of neutrinos momentum $\mathbf{q}$ at position $\mathbf{r}$, in other
words the number of particles in the phase-space volume $\mathrm{d} \mathbf{q} \mathrm{d} \mathbf{r}$.

It is common to work in terms of the specific intensity $\mathfrak{I}(\mathbf{r},\hat{n},\xi,t)$,
which is defined such that the amount of energy $d\xi$ transported in the energy interval
$(\xi,\xi+\mathrm{d}\xi)$, by neutrinos propagating into 
the solid angel $\mathrm{d}\Omega$ in the direction $\hat{n}$,
through a surface with area $\mathrm{d}A$, with normal vector $\hat{A} = \mathbf{r}/|\mathbf{r}|$, 
during the time interval $\mathrm{d}t$ is
\begin{equation} \label{eqN:intns}
d \xi = \mathfrak{I}(\mathbf{r},\hat{n},\xi,t) \, \hat{n} \cdot \hat{A}
\, \mathrm{d}\xi \, \mathrm{d}A \, \mathrm{d}\Omega \, \mathrm{d}t.
\end{equation}
The neutrino distribution function and the specific neutrino intensity is related as follows
\begin{equation}
\mathfrak{I}(\mathbf{r},\hat{n},\xi,t) = \frac{\xi^3}{h^3 c^2} f(\mathbf{r},\hat{n},\xi,t).
\end{equation}
The evolution of the specific neutrino intensity is calculated by solving the Boltzmann equation
\begin{equation} \label{eqN:boltz}
\frac{1}{c} \partial_t \mathfrak{I} + \hat{n}_i \partial_i \mathfrak{I} = \mathfrak{C} [\mathfrak{I}].
\end{equation}
The right hand side of \eq{eqN:boltz} is a source term that describes scattering, emission and absorption
of neutrinos (the so-called collision integral). $\mathfrak{C} [\mathfrak{I}]$ will in general depend on integrals of the specific neutrino intensity, witch makes the problem difficult to solve numerically. 
A common strategy is to expand the specific neutrino intensity into angular moments
\begin{alignat}{2}
\mathfrak{L}(\mathbf{r},\xi,t) &\equiv  \frac{1}{4\pi} \int \mathfrak{I} \, \mathrm{d} \Omega &&\qquad (\nth{0}-order), \\ 
\mathfrak{H}_i(\mathbf{r},\xi,t) &\equiv  \frac{1}{4\pi} \int \mathfrak{I} \, \hat{n}_i \, \mathrm{d} \Omega &&\qquad (\nth{1}-order), \\
\mathfrak{K}_{ij}(\mathbf{r},\xi,t) &\equiv  \frac{1}{4\pi} \int \mathfrak{I} \, \hat{n}_i \, \hat{n}_j \, \mathrm{d} \Omega &&\qquad (\nth{2}-\text{order}), \\
& \ \ \vdots && \nonumber
\end{alignat}
and then solve the equations that arises when inserting these moments into \eq{eqN:boltz}
\begin{align}
\frac{1}{c} \partial_t \mathfrak{L} +  \partial_i \mathfrak{H}_i &= \frac{1}{4\pi} \int \mathfrak{C} [\mathfrak{I}] \mathrm{d} \Omega, \\
\frac{1}{c} \partial_t \mathfrak{H}_i + \partial_i \mathfrak{K}_{ij} &= \frac{1}{4\pi} \int \mathfrak{C} [\mathfrak{I}] \hat{n}_i \mathrm{d} \Omega. \\
& \ \ \vdots \nonumber
\end{align}
We see from these equations that the evolution of the $k^{\text{th}}$-order moment depends on $(k+1)^{\text{th}}$-order moment.

The version of \textsc{Vertex} that is implemented into \textsc{Prometheus-Vertex} uses
the so-called``ray-by-ray-plus'' approximation of \cite{buras_06a} (see \cite{hanke_phd} for details about the implementation).
For each angular direction of the simulation grid a spherical symmetric radiation problem is solved
with the ``ray-by-ray-plus'' method, by assuming that the radiation field is symmetric around the rays propagation direction.
In other words, the code traces one ``ray'' per angular bin, this is what is known as the ``ray-by-ray'' method.
In this case the angular moments of the specific neutrino intensity can be represented by scalars.
Furthermore, the expansion of the specific neutrino intensity is truncated at $\nth{1}$-order and
closed with a variable Eddington factor method. The two Eddington factors that are needed to close the system are the ratio of
the $\nth{0}$-order and the $\nth{3}$-order scalar angular moments and the ratio of the $\nth{0}$-order and the $\nth{4}$-order scalar angular moments. The two factors are calculated from a simplified version of the Boltzmann equation in an iterative process until convergence within an acceptable error is reached. 
\cite{buras_06b} found that it is necessary to the take non-radial advection of neutrinos and 
the non-radial neutrino pressure gradients into account to avoid un-physical convection in the PNS. 
The inclusion of these terms is what is meant by the ``plus'' in ``ray-by-ray plus''.

\textsc{Vertex} solves the neutrino transport problem for three neutrino species, $\nu_e$, $\bar{\nu}_e$, and a species $\nu_X$ representing
all heavy flavor neutrinos. 
For the models that make the basis of this work, the neutrinos were binned into 12 logiartimicly spaced energy bins ranging from 0 to 380 MeV.  
A more detail description of the numerical implementation, neutrino physics, and the \textsc{Vertex} in general can
be found \cite{rampp_02}, \cite{hanke_phd}, and \cite{melson_phd}.

\section{Grid setup}
Since stars are, at least to lowest order, spherical objects it is advantagous to use a spherical grid when performing numerical simulations.
\textsc{Prometheus-Vertex} can use two different spherical grids.

\subsection{Spherical polar grid}
The first grid available in \textsc{Prometheus-Vertex} is a standard spherical polar grid, 
with mesh points $r_n \in [0, R]$, $\theta_n \in [0, \pi]$, and $\phi_n \in [0, 2\pi]$.
The grid is logarithmic space in radius and the angular points are spaced equidistantly.  
The structure of ``standard'' spherical polar $(r, \theta, \phi)$ has a few shortcomings that
can be problematic when performing numerical simulations. 
The grid contains coordinate singularities at the poles ($\theta = 0$ and $\theta = \pi$)
and grid cells become smaller and smaller the closer one gets we get to the poles. This causes strong constraints on the time step and
can even lead to numerical artifacts near the poles \citep{wongwathanarat_10a,mueller_15b}.

\subsection{Yin-Yang grid}
To avoid the problems encountered at the poles of a standard spherical grid, \cite{kageyama_04} proposed to
construct a grid from two geometrically identical subgrids (called Yin and Yang). 
The two sub-grids are both spherical and have identical local coordinates, their mesh points are defined as follows
\begin{align}
r_n^{Y} &\in [0, R], \\
\theta_n^{Y} &\in [pi/4, 3\pi/4], \\
\phi_n^{Y} &\in [-3pi/4, 3\pi/4].
\end{align} 
Here the superscript $Y$ referrers to either the \textit{Yin} or \textit{Yang}. While the two grids
have identical local coordinates, they are rotated in respect to each other in such a way
to cover the whole sphere. The two grids are orientated in such that if 
the Cartesian coordinates of the \textit{Yin} grid are $(x^{Yin},y^{Yin},z^{Yin})$
then the local Cartesian coordinates of the \textit{Yang} grid are
\begin{equation}
(x^{Yang},y^{Yang},z^{Yang}) = (-x^{Yin},z^{Yin},y^{Yin}).
\end{equation} 
The \textit{Yang} grid is rotated 90 degrees around the x-axis
and 180 degrees around the y-axis of the \textit{Yin} grid. 
This corresponds to the rotation matrix
\begin{equation} \label{eqT:pij}
R = 
  \begin{pmatrix}
    -1 & 0 & 0  \\
    0 & 0 & 1 \\
    0 & 1 & 0
  \end{pmatrix}.
\end{equation}

With this prescription we avoid the problems that arise
near the poles of a normal spherical grid.
For more details about the implementation of the \textit{Yin}-\textit{Yang} grid
in \textsc{Prometheus-Vertex} see \cite{melson_phd}.  

% \begin{tikzpicture}
% \foreach \t in {0,10,...,180}
%     {\draw[black] ({2*cos(\t)},{2*sin(\t)},0)
%          \foreach \rho in {5,10,...,360}
%              {--({2*cos(\t)*cos(\rho)},{2*sin(\t)*cos(\rho)},
%          {2*sin(\rho)})}--cycle;
%     }
% \foreach \t in {-90,-85,...,90}% parallels
%         {\draw[black] ({2*cos(\t)},0,{2*sin(\t)})
%      \foreach \rho in {5,10,...,360}
%      {--({2*cos(\t)*cos(\rho)},{2*cos(\t)*sin(\rho)},
%              {2*sin(\t)})}--cycle;
%     } 
% \end{tikzpicture}
% \begin{tikzpicture}
% \foreach \t in {45,55,...,135}
%     {\draw[black] ({2*cos(\t)},{2*sin(\t)},0)
%          \foreach \rho in {-135,-125,...,135}
%              {--({2*cos(\t)*cos(\rho)},{2*sin(\t)*cos(\rho)},
%          {2*sin(\rho)})}--cycle;
%     }
% % \foreach \t in {-90,-85,...,90}% parallels
% %         {\draw[black] ({2*cos(\t)},0,{2*sin(\t)})
% %      \foreach \rho in {5,10,...,360}
% %      {--({2*cos(\t)*cos(\rho)},{2*cos(\t)*sin(\rho)},
% %              {2*sin(\t)})}--cycle;
% %     } 
%\end{tikzpicture}
