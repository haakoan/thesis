
%%% Local Variables:
%%% mode: latex
%%% TeX-master: "../../doktorarbeit"
%%% End:
\chapter{Numerical simulations of core-collapse supernovae}
The GW signals presented in this thesis is based on 
supernova simulations that was performed with the code \textsc{Prometheus-Vertex}.
The simulations were performed by members of the Garching-group 
(\citealp{hanke_phd}, \citealp{melson_phd} and \citealp{suma_models}).
In this chapter we described the code used to carry out these simulations, we will
describe the details of individual models in later chapters. 
\textsc{Prometheus-Vertex} solves the combined problem of hydrodynamics and neutrino radiation transport.
The code has been specifically developed for the core-collapse problem and has been developed for over
decade, the first version was developed in 2002 by \cite{rampp_02} 

\section{\textsc{Prometheus}}
\subsection{Hydrodynamics}
The hydrodynamics of the stellar collapse is solved with the with a version of the
well-established \textsc{Prometheus} code \citep{mueller_91,fryxell_91}. It solves the non-relativistic 
equations of hydrodynamics in spherical coordinates $(r,\theta,\phi)$.
In spherical coordinates the Euler reads as follows
\begin{alignat}{1}
&\partial_t \rho + \frac{1}{r^2} \partial_r (r^2 \rho v_r) + \frac{1}{r \sin{\theta}} \bigg[ \partial_{\theta} (\rho \sin{\theta} v_{\theta}) + \partial_{\phi} (\rho v_{\phi}) \bigg] = 0, \label{eqN:cont} \\
\stepcounter{equation}
&\partial_t (\rho v_r) + \frac{1}{r \sin{\theta}} \bigg[ \partial_{\theta} (\rho \sin{\theta} v_{\theta} v_r)
+\partial_{\phi} (\rho v_{\phi} v_r) \bigg]  + \nonumber \\ 
&\frac{1}{r^2} \partial_r (r^2 \rho v_r^2)  - \rho \frac{v_{\theta}^2 + v_{\phi}^2}{r} + \partial_r p 
= - \rho \partial_r \Phi + Q_{M_r}, \tag{\theequation{}a} \\
%
&\partial_t (\rho v_{\theta}) + \frac{1}{r^2} \partial_r (r^2 \rho v_r v_{\theta}) + \frac{1}{r \sin{\theta}}\bigg[ \partial_{\theta} (\rho \sin{\theta} v_{\theta}^2) +
\partial_{\phi} (\rho v_{\phi} v_{\theta}) \bigg] \nonumber \\
&+\rho \frac{v_{\theta}v_r - v_{\phi}^2/\tan{\theta}}{r} + \frac{1}{r}\partial_{\theta} p = - \frac{\rho}{r} \partial_{\theta} \Phi + Q_{M_{\theta}}, \tag{\theequation{}b} \\
%
&\partial_t (\rho v_{\phi})  
+\frac{1}{r \sin{\theta}} \bigg[ \partial_{\theta} (\rho \sin{\theta} v_{\theta} v_{\phi}) +
\partial_{\phi} (\rho v_{\phi}^2)  + \partial_{\phi} p  \bigg] \nonumber \\
&+ \frac{1}{r^2} \partial_r (r^2 \rho v_r v_{\phi}) +
\rho \frac{v_{\theta}v_{\phi}/\tan{\theta} + v_{\phi}v_r}{r}
 = Q_{M_{\phi}}, \tag{\theequation{}c} \\
%
\stepcounter{equation}
&\partial_t e + \frac{1}{r^2} \partial_r (r^2 v_r (e + p))
+\frac{1}{r \sin{\theta}} \bigg[ \partial_{\theta} ( \sin{\theta} v_{\theta} (e + p))  +
\partial_{\phi} (v_{\phi} (e + p)) \bigg] \nonumber \\
&= -\rho v_r \partial_r \Phi - \rho\frac{v_{\theta}}{r}\partial_{\theta}\phi
 + Q_E + v_r Q_{M_r} + v_{\theta} Q_{M_{\theta}} + v_{\phi} Q_{M_{\phi}}.
\end{alignat}
Here $p$ is pressure, $\rho$ is density, $e$ is the specific total energy,  $\Phi$ is the gravitational potential and
$(v_r,v_{\theta}, v_{\phi}$ are the velocity components in the spherical
coordinate systems. The terms $Q_{M_{r}}$, $Q_{M_{\theta}}$, $Q_{M_{\phi}}$ represents
the neutrino source terms for momentum transfer, in the radial, polar, and azimuthal direction, respectively.
Energy transport by neutrinos is represent by the source term $Q_{M_{E}}$. These source terms are calculated
by the neutrino-transport module \textsc{Vertex} and we will come back to them later in the chapter.
The above equations has to be closed by a EoS, which generally will depend on 
density, internal energy, and the chemical composition of the stellar matter.
This means that one has to track of two additional quantities, the mass fractions
of different nuclear species, denoted by $X_i$, and the electron fraction $Y_e$. This means
that two additional conservation equations has to be solved:
\begin{alignat}{1}
\partial_t (\rho X_i) + \frac{1}{r^2} \partial_r (r^2 \rho v_r X_i) + \frac{1}{r \sin{\theta}} \bigg[ \partial_{\theta} (\rho \sin{\theta} v_{\theta} X_i) + \partial_{\phi} (\rho v_{\phi} X_i) \bigg] &= \varsigma_i, \label{eqN:contxi} \\
\partial_t (\rho Y_e) + \frac{1}{r^2} \partial_r (r^2 \rho v_r Y_e) + \frac{1}{r \sin{\theta}} \bigg[ \partial_{\theta} (\rho \sin{\theta} v_{\theta} Y_e) + \partial_{\phi} (\rho v_{\phi} Y_e) \bigg] &= Q_{Y_e}. \label{eqN:contye} \\
\end{alignat}
The two terms $\varsigma_i$ and $Q_{Y_e}$ represents the change of composition of species $i$ due nuclear reactions
and the change in electron fraction caused by emission and absorption of electron and anti-electron neutrinos,
respectively. If the fluid reaches nuclear statistical equilibrium (NSE), then the chemical 
composition is fully determined by the equation of state, through the electron fraction, density, and
temperature.

\textsc{Prometheus} solves the system of equations described in the paragraph above
by means of a dimensionally-split implementation of the piece-wise parabolic method of \cite{colella_84}.
The scheme is time-explicit and it is accurate to third-order in space and second-order in time.
The Riemann solver exactly solves 1D Riemann problems in so-called sweeps
that have been obtained from the full 3D equations by Strange-splitting \citep{strang_68}.
When strong shocks are encountered the solver switches to the ``HLLE'' solver \citep{einfeldt_88}.
This is done in order to avoid the so-called ``even-odd decoupling'' which occurs
when the shocks are aligned with one of the axis of the simulation grid \citep{quirk_94,kifonidis_03}
and creates artificial oscillations of the shock. \textsc{Prometheus} employs the 
consistent multi-fluid advection method of \cite{plewa_99} to ensure that the advection all the nuclear species
are calculated accurately.

\subsection{Equation of state}
The models that this work is based on uses two different prescriptions
for the EoS. A ``high-density'' EoS is used for the inner
hot region of the core, while a ``low-density'' EoS is used for the low-density
parts of the simulation volume. The two are separated by a density threshold,
after core bounce this threshold is set to $\rho_T = 10^{11}$ g/cm$^3$.
For the high-density regime the tabular EoS of 
\cite{lattimer_91} with a nuclear incompressibility of 220 MeV is used. 
Below the threshold, in the low-density regime, a EoS that describes the nuclei as
classical Boltzmann gases. Electrons and positrons are described as Fermi gases with 
arbitrary degeneracy levels. The EoS also include the effect of photons \citep{janka_99}.

\subsection{Self gravity - the effective potential}
Self-gravity is treated using the monopole                      
approximation, and the effects of general relativity are accounted for                                               
in an approximate fashion by means of a pseudo-relativistic                                                          
effective potential (case~A of \citealt{marek_06}). 
The effective potential includes general relativistic effects and
take the properties of the medium into account, such as the pressure and energy
of the fluid elements. 

\section{\textsc{Vertex}}
The \textsc{Vertex} code calculates the source terms on the right hand side of the
Euler equations by treating the neutrinos as a radiation field, which is justified
because the mean-free path of the neutrinos is typically much greater than the
typical size of a fluid element. The problem at hand is to describe 
the distribution of the neutrinos in position ($\mathbf{r}$) and momentum ($\mathbf{q}$),
in other words in the six-dimensional phase-space described by $\mathbf{r}$ and $\mathbf{q}$,
and how this distribution evolves in time.
That is to determine the neutrino distribution function $f(\mathbf{r},\mathbf{q},t)$.
It is, however, common to work in terms of the specific intensity $\mathfrak{I}$,
which is defined such that the amount of energy $d\xi$ transported in the energy interval
$(\xi,\xi+\mathrm{d}\xi)$, by neutrinos propagating into 
the solid angel $\mathrm{d}\Omega$ in the direction $\hat{n}$,
through a surface with area $\mathrm{d}A$, with normal vector $\hat{A} = \mathbf{r}/|\mathbf{r}|$, 
during the time interval $\mathrm{d}t$ is
\begin{equation} \label{eqN:intns}
d \xi = \mathfrak{I}(\mathbf{r},\hat{n},\xi,t) \, \hat{n} \cdot \hat{A}
\, \mathrm{d}\xi \, \mathrm{d}A \, \mathrm{d}\Omega \, \mathrm{d}t.
\end{equation}
The neutrino distribution function and the specific neutrino intensity is related as follows
\begin{equation}
\mathfrak{I} = \frac{\xi^3}{h^3 c^2} f.
\end{equation}
The evolution of the specific neutrino intensity is calculated by solving the Boltzmann equation
\begin{equation} \label{eqN:boltz}
\frac{1}{c} \partial_t \mathfrak{I} + \hat{n}_i \partial_i \mathfrak{I} = \mathfrak{C} [\mathfrak{I}].
\end{equation}
The right hand side of \eq{eqN:boltz} is a source term that describes scatter, emission and absorption
of neutrinos (the so-called collision integral). $\mathfrak{C} [\mathfrak{I}]$ will generally depend on integrals of the specific neutrino intensity, witch makes it very difficult to solve numerically. 
A common strategy is to expand the specific neutrino intensity into angular moments,
\begin{alignat}{2}
\mathfrak{J}(\mathbf{r},\xi,t) &\equiv  \frac{1}{4\pi} \int \mathfrak{I} \, \mathrm{d} \Omega &&\qquad (\nth{0}-order), \\ 
\mathfrak{H}_i(\mathbf{r},\xi,t) &\equiv  \frac{1}{4\pi} \int \mathfrak{I} \, \hat{n}_i \, \mathrm{d} \Omega &&\qquad (\nth{1}-order), \\
\mathfrak{K}_{ij}(\mathbf{r},\xi,t) &\equiv  \frac{1}{4\pi} \int \mathfrak{I} \, \hat{n}_i \, \hat{n}_j \, \mathrm{d} \Omega &&\qquad (\nth{2}-\text{order}), \\
& \ \ \vdots && \nonumber
\end{alignat}
and then solve the equations that arises when inserting these moments into \eq{eqN:boltz}
\begin{align}
\frac{1}{c} \partial_t \mathfrak{J} +  \partial_i \mathfrak{H}_i &= \frac{1}{4\pi} \int \mathfrak{C} [\mathfrak{I}] \mathrm{d} \Omega, \\
\frac{1}{c} \partial_t \mathfrak{H}_i + \partial_i \mathfrak{K}_{ij} &= \frac{1}{4\pi} \int \mathfrak{C} [\mathfrak{I}] \hat{n}_i \mathrm{d} \Omega. \\
& \ \ \vdots \nonumber
\end{align}
We see from these equations that the evolution of the $k^{\text{th}}$-order moment depends on $(k+1)^{\text{th}}$-order moment.

The version of \textsc{Vertex} that is implemented into \textsc{Prometheus-Vertex} uses
the so-called``ray-by-ray-plus'' approximation of \cite{buras_06a} (see \cite{hanke_phd} for details about the implementation),
in which it is assumed that the specific intensity is axissymmetric around the radial direction. 
In this case the angular moments of the specific neutrino intensity can be represented by scalars.
Furthermore, the expansion of the specific neutrino intensity is truncated at $\nth{1}$-order and
closed with a variable Eddington factor method. The two Eddington factors that are needed to close the system are the ratio of
the $\nth{0}$-order and the $\nth{3}$-order scalar angular moments and the ratio of the $\nth{0}$-order and the $\nth{4}$-order scalar angular moments.
The two factors are calculated from a simplified version of the Boltzmann equation in an iterative 
process until they converge. 

\textsc{Vertex} solves equations for three neutrino species, $\nu_e$, $\bar{\nu}_e$, and a species $\nu_X$ representing
all heavy flavor neutrinos. For each radial direction of the simulation grid one the spherical symmetric radiation problem is solved.
In other words the code traces one ``ray'' per angular bin, this is what is known as the ``ray-by-ray'' method. \cite{buras_06b} 
found that it is necessary to the take non-radial advection of neutrinos and the non-radial neutrino pressure gradients into account to avoid 
un-physical convection in the PNS. The inclusion of these terms is what is meant by the ``plus'' in ``ray-by-ray plus''.
For the models that makes the basis of this work the neutrinos were binned into 12 logiartimicly spaced energy bins ranging from 0 to 380 MeV.  
A more detail description of the numerical implementation, neutrino physics, and the \textsc{Vertex} in general can
be found \cite{rampp_02}, \cite{hanke_phd}, and \cite{melson_phd}.
