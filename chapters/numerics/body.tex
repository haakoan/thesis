
%%% Local Variables:
%%% mode: latex
%%% TeX-master: "../../doktorarbeit"
%%% End:
\chapter{Numerical simulations of core-collapse supernovae}
The GW signals presented in this thesis is based on 
supernova simulations that was performed with the code \textsc{Prometheus-Vertex}.
The simulations were performed by members of the Garching-group 
(\citealp{hanke_phd}, \citealp{melson_phd} and \citealp{suma_models}).
In this chapter we described the code used to carry out these simulations and 
we describe the individual models in later chapters. 
\textsc{Prometheus-Vertex} solves the combined problem of hydrodynamics and neutrino radiation transport.
The code has been specifically developed for the core-collapse problem and has been developed for over
decade, the first version was developed in 2002 by \cite{rampp_02} 
\section{\textsc{Prometheus}}
The hydrodynamics of the stellar collapse is solved with the with a version of the
well-established  \textsc{Prometheus} code. It solves the non-relativistic 
equations of hydrodynamics in spherical coordinates $(r,\theta,\phi)$.

\begin{alignat}{2}
&\partial_t \rho + \frac{1}{r^2} \partial_r (r^2 \rho v_r) + \frac{1}{r \sin{\theta}} \bigg[ \partial_{\theta} (\rho \sin{\theta} v_{\theta}) +
\partial_{\phi} (\rho v_{\phi}) \bigg] &&= 0 \\
%
\stepcounter{equation}
&\partial_t (\rho v_r) + \frac{1}{r \sin{\theta}} \bigg[ \partial_{\theta} (\rho \sin{\theta} v_{\theta} v_r)
+\partial_{\phi} (\rho v_{\phi} v_r) \bigg]&&\nonumber \\ 
& + \frac{1}{r^2} \partial_r (r^2 \rho v_r^2)  - \rho \frac{v_{\theta}^2 + v_{\phi}^2}{r} + \partial_r p 
&&= - \rho \partial_r \Phi + Q_{M_r} \tag{\theequation{}a} \\
%
&\partial_t (\rho v_{\theta}) + \frac{1}{r \sin{\theta}}\bigg[ \partial_{\theta} (\rho \sin{\theta} v_{\theta}^2) +
\partial_{\phi} (\rho v_{\phi} v_{\theta}) \bigg] &&\nonumber \\  
& + \frac{1}{r^2} \partial_r (r^2 \rho v_r v_{\theta})
+\rho \frac{v_{\theta}v_r - v_{\phi}^2/\tan{\theta}}{r} + \frac{1}{r}\partial_{\theta} p 
&&= - \frac{\rho}{r} \partial_{\theta} \Phi + Q_{M_{\theta}} \tag{\theequation{}b} \\
%
&\partial_t (\rho v_{\phi})  
+\frac{1}{r \sin{\theta}} \bigg[ \partial_{\theta} (\rho \sin{\theta} v_{\theta} v_{\phi}) +
\partial_{\phi} (\rho v_{\phi}^2)  + \partial_{\phi} p  \bigg] &&\nonumber \\
&+ \frac{1}{r^2} \partial_r (r^2 \rho v_r v_{\phi}) +
\rho \frac{v_{\theta}v_{\phi}/\tan{\theta} + v_{\phi}v_r}{r}
&& = Q_{M_{\phi}} \tag{\theequation{}c} \\
%
&\partial_t e + \frac{1}{r^2} \partial_r (r^2 v_r (e + p)) && \nonumber \\
& +\frac{1}{r \sin{\theta}} \bigg[ \partial_{\theta} ( \sin{\theta} v_{\theta} (e + p))  +
 \partial_{\phi} (v_{\phi} (e + p)) \bigg] &&= -\rho v_r \partial_r \Phi - \rho\frac{v_{\theta}}{r}\partial_{\theta}\phi \nonumber \\
& && + Q_E + v_r Q_{M_r} \nonumber \\
& && + v_{\theta} Q_{M_{\theta}} + v_{\phi} Q_{M_{\phi}}.
\end{alignat}



% which is a Newtonian finite-volume solver
% for the conservation equations (2.1), (2.3), and (2.5). Prometheus solves hydrody-
% namic problems on orthogonal grids in spherical symmetry (1D), axial symmetry (2D),
% or full three dimensions (3D) with high numerical accuracy.











% \comment The simulations were performed with the \textsc{Prometheus-Vertex} code \citep{rampp_02,buras_06a}. 
% The Newtonian hydrodynamics module
% \textsc{Prometheus} \citep{mueller_91,fryxell_91} features a dimensionally-split
% implementation of the piecewise parabolic method of \cite{colella_84}
% in spherical polar coordinates $(r,\theta,\varphi)$. Self-gravity is treated using the monopole
% approximation, and the effects of general relativity are accounted for
% in an approximate fashion by means of a pseudo-relativistic
% effective potential (case~A of \citealt{marek_06}). The neutrino transport
% module \textsc{Vertex} \citep{rampp_02} solves the energy-dependent two-moment
% equations for three neutrino species ($\nu_e$, $\bar{\nu}_e$, and a species $\nu_X$ representing
% all heavy flavor neutrinos) using a variable Eddington factor technique.
% The `ray-by-ray-plus'' approximation of \citet{buras_06a} is applied to make the
% multi-D transport problem tractable. In the high-density regime, the nuclear equation of  state (EoS) of \citet{lattimer_91} with a bulk incompressibility modulus of nuclear matter of $K=220 \,\mathrm{MeV}$
% has been used in all cases.