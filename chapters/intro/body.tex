%%% Local Variables:
%%% mode: latex
%%% TeX-master: "../../doktorarbeit"
%%% End:
\chapter{Introduction}
We recently experienced the first direct detection of gravitational waves \citep{gw_detect},
which marks the beginning of the era of gravitational wave astronomy. Gravitational waves can provide us with 
unique insights where traditional photon based astronomy fails. An example of such a case is core-collapse supernovae. 
Electromagnetic radiation created in the core of a massive star during collapse is absorbed by the stellar
envelope and we are not able to directly probe the workings of a core with 
photometric observations. Gravitational waves, and also neutrinos, on the other hand, can
propagate freely through the material surrounding the core and could, if detected, provide us with
direct information about the nature of core-collapse supernovae.

In this thesis, we will analyse the gravitational wave signals
from seven three-dimensional simulations of core-collapse supernovae. 
The goal is to connect features in the signal
to physical processes taking place in the simulations. We will compare the results to
those of two-dimensional studies and analyse the differences between previous studies and our results in detail.
The main results of this work will be discussed in chapters~\ref{ch:paper1}, and ~\ref{ch:paper2}. 
Both chapters are led by an introduction that discusses the relevant literature and motivates the work presented in
the corresponding chapter.
In chapter~\ref{ch:paper1} we study four non-rotating progenitors and focus on the
underlying hydrodynamic instabilities responsible for gravitational wave emission,
we will also dedicate a large portion of this chapter to discussing the differences between
the signals presented in this thesis and those from two-dimensional models found in the literature.
Then, in chapter~\ref{ch:paper2} we will discuss how slow to moderate progenitor rotation influences the emission of
gravitational waves. In both chapters, we assess the detection prospects of the models.

Before we discuss the gravitational wave signals, we will go through the basics theory of gravitational waves in
chapter~\ref{ch:theory}, here we will describe how the gravitational wave signal is extracted from numerical simulations. 
In chapter~\ref{ch:ccsn} we outline the current core-collapse paradigm. The numerical
methods used to simulated the models are discussed in chapter~\ref{ch:numerics}. Then, in chapter~\ref{ch:signal}
we go through the tools and techniques used to analyse the signal. The next two chapters are dedicated to
the actual study of the gravitational waves from the simulations. In the last chapter, we will summarise and discuss
the results, before giving a short outlook of the field.
\newpage
\vspace*{\fill}
\begingroup
\centering
In short, we will attempt to find the 
complicated answer to the simple question:
\begin{displayquote}
\textit{What can we learn about the explosion mechanism of core-collapse supernovae
from observing the gravitational waves they emit?}
\end{displayquote}
\endgroup
\vspace*{\fill}
