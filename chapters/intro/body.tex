%%% Local Variables:
%%% mode: latex
%%% TeX-master: "../../doktorarbeit"
%%% End:
\chapter{Introduction}
The subject of this thesis can be summarised by one simple question:
\begin{displayquote}
\textit{What can we learn about the explosion mechanism of core-collapse supernovae
from observing the gravitational waves they emit?}
\end{displayquote}
Like most questions worth asking, the answer is not as easy to formulate 
as the question itself. There are several factors that will impact the 
answer, and each of these factors add a new layer of complexity.

To give a short, and optimistic answer, to the question, let us assume
that a nearby star, for example Spica (10.2 \msun) of the Virgo constellation 
or Betelgeuse (11.6 \msun) in Orion, went supernova some few hundred years ago.
The gravitational radiation from a supernova that close would be very strong and
we would be able to accurately recreate the waveforms. In this case we could learn a lot.
The advantage of gravitational waves, and neutrinos, compared to electromagnetic radiation
is that they allow us to probe the inner core of the star. The photons emitted by the
central core are absorbed as they travel through the star, not so for gravitational radiation.
Being able to see directly into the core of the star would provides us with a wealth of information
that would undoubtedly increase our understanding of exactly how massive stars explode. Gravitational
waves are sensitive to the mass motions within the core and can provide accurate information about
the dynamics of the explosion.

Even this optimal case is complicated by the fact that we would have to 
attribute characteristics of the signal to underlying physical processes, which
requires a good theoretical understanding of the core-collapse of massive stars.
Our current, but still somewhat lacking, understanding of ordinary core-collapse
supernovae (supernovae with energies up to $\sim 10^{51}$ erg) is that the explosion
is powered by the so called delayed neutrino-driven mechanism \citep{wilson_85}, 
which we will discuss in detail in chapter~\ref{ch:ccsn}. The analytic and numerical efforts
of the last few decades have recently cumilated in the emergence of the first sucessfull
fully three-dimensional simulations of supernovae explosions \citep{melson_15a,melson_15b,lentz_15, suma_models}.

In this thesis we will make predictions for and analyse the gravitational wave signals
from such numerical simulations. We will study a wide range of both sucessfull and unscussfull
three-dimensional simulations. The goal of the thesis is to connect fetures in the signal
to physical processes taking place in the simulations. We will compare the results to
those of two-dimensional studies and analyse the differences between the two in detail.
In chapter~\ref{ch:paper1} we study four non-rotating progenitors and focus on the
underlying hydrodynamic instabilites responsible for gravitational wave emission,
we will also dedicate a large portion of this chapter to discussing the diffrences between
the signals presented in this thesis and those from two-dimensional models found in the litrature.
Then, in chapter~\ref{ch:rot} we will discuss how slow to moderate progenitor rotation influences the emission of
gravitational waves. In both chapters, we asses the detection prospects of the models.

Before we discuss the gravitational wave signals, we will go through the basics of gravitational waves in
chapter~\ref{ch:gwtheory}, here we will also describe how the gravitational wave signal is extracted from the numerical simulations. In chapter~\ref{ch:ccsn} outline the current core-collapse paradigem. The numerical
methods used to simulated the models is disucssed in chapter~\ref{ch:numerics}. Then, in chapter~\ref{ch:signal}
we go through the tools and tequenices used to analyse the signal. The next two chapers, are dedicated to
the actual study of the gravitational waves from the simulations. In the last chapter we will sumarise and disscuss
the results, before giving a short outlook of the field.
