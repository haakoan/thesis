%%% Local Variables:
%%% mode: latex
%%% TeX-master: "../../doktorarbeit"
%%% End:
\chapter{Signal analysis}
\section{Fourier analysis}
In order to study the frequency structure of a time signal it is common 
to approximate the signal as a sum of trigometric functions. Two examples
of such analysis is Fourier seris and Fourier transforms, we will mainly
focus on the latter.
\subsection{Continuous and discrete Fourier transforms}
From a continuous time signal $x(t)$ we can define the continuous Fourier transform  
\begin{equation} \label{eqSA:cf}
\widetilde{x}(f) = \int_{-\infty}^{\infty} x(t) e^{-2 \pi i f t} \ud t, 
\end{equation}
that transforms the function into the frequency ($f$) domain.
Simply put, the Fourier transformation tells us how much a harmonic oscillation 
of frequency $f$ contributes to the total signal. 

If the time signal is represented by an discrete time series, as is the case with data from simulations,
the integral on the right hand side of \eq{eqSA:cf} can be estimated by numerical integration and 
we can derive an expression for the correspodning discrete Fourier transform (DFT).
Consider the time series $x_m$ of duration $T$ which has been obtained by sampling the underlying 
continuous signal $x(t)$ at $M$ discrete evenly spaced times
\begin{equation}
x_m = x(t_m) \qquad m = 0,1,2,3,4 ... M.
\end{equation} 
Since the sampeling is evenly spaced we have that 
\begin{equation}
t_m = m\Delta t \qquad m = 0,1,2,3,4 ... M,
\end{equation} 
where $\Delta t( = T/M)$ is the sampeling interval.
Since we have chosen to sample our function at $M$ times over a time period
$T$ we have by construction a function which has period $T$, regardless of the true periodicity of the
underlying function. This implies that $x_M = x_0$. Since our signal has period $T$ it means that frequency
of the slowest varying ossilation we can messure is $1/T$. By the same logic the second slowest ossilation 
we can capture has a frequency of $1/(T-\Delta t)$ and so on. This means that the discrete signal
is represented by a set of $N+1$ frequencies 
\begin{equation}
f_k = k/M \Delta t \qquad k = 0,1,2,3,4 ... M.
\end{equation}
We now calculate the integral in \eq{eqSA:cf} for a given frequency $f_k$ using the trapzoidal rule and find
\begin{align}
\int_{-\infty}^{\infty} x(t) e^{-2 \pi i f_k t} \ud t &\simeq \int_{0}^{T} x(t) e^{-2 \pi i f_k t} \ud t \nonumber \\
&\approx \sum_{m=1}^{M} x_m e^{-2 \pi i f_k t_m} \Delta t \nonumber \\
& = \sum_{m=1}^{M} x_m e^{-2 \pi i k m/M} \Delta t \equiv \widetilde{X}_k \Delta t M.
\end{align}
In the last line we have defined the discrete Fourier transform (of a time series $x_m$)
\begin{equation} \label{eq:DFT}
\widetilde{X}_k (f_k) = \frac{1}{M}  \sum^M_{m=1} x_m e^{-2\pi i k m/M}.
\end{equation}

\section{The short-time Fourier transform}
The dissatvangate of the Fourier transform is that it only gives you information
about the frequency spectrum of the full time signal. If we are analysing a 
signal with a varying frequency structure it will be useful to extract 
time-frequency information about the signal. This is done by separating
the signal into shorter segments and then calculating the Fourier transform of each
segment.
