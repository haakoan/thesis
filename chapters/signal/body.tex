%%% Local Variables:
%%% mode: latex
%%% TeX-master: "../../doktorarbeit"
%%% End:
\chapter{Signal analysis}
\section{Fourier analysis}
In order to study the frequency structure of a time signal it is common 
to approximate the signal as a sum of trigonometric functions. Two examples
of such analysis is Fourier series and Fourier transforms, in this chapter we will mainly
focus on the latter and discuss the tools used for signal analysis in this thesis.
\subsection{Continuous and discrete Fourier transforms}
We define the continuous Fourier transform (FT) of a continuous time signal $x(t)$ as  
\begin{equation} \label{eqSA:cf}
\widetilde{x}(f) = \int_{-\infty}^{\infty} x(t) e^{-2 \pi i f t} \ud t. 
\end{equation}
The FT transforms the signal from the time domain into the frequency ($f$) domain.
Simply put, the Fourier transformation tells us how much an oscillation of frequency $f$ contributes to the total signal. 

If the time signal is represented by a discrete time series, as is the case with data from simulations,
the integral on the right-hand side of \eq{eqSA:cf} can be estimated by numerical integration and 
we can derive an expression for the corresponding discrete Fourier transform (DFT).
Consider the time series $x_m$ of duration $T$ which has been obtained by sampling the underlying 
continuous signal $x(t)$ at $M$ discrete, evenly spaced, times
\begin{equation}
x_m = x(t_m) = x(m \, \Delta t) \qquad m = 0,1,2,3,4 ... M,
\end{equation} 
where $\Delta t( = T/M)$ is the sampling interval.
The time series $x_m$ is by construction a periodic function with period $T$, regardless of the true periodicity of the
underlying function. This implies that $x_M = x_0$. Since our signal has period $T$ it means that frequency
of the slowest varying oscillation we can represent is $1/T$. By the same logic the second slowest oscillation 
we can capture has a frequency of $1/(T-\Delta t)$ and so on. This means that the discrete signal
is represented by a set of $M+1$ frequencies 
\begin{equation}
f_k = k/M \Delta t \qquad k = 0,1,2,3,4 ... M.
\end{equation}
We now calculate the integral in \eq{eqSA:cf} for a given frequency $f_k$ using the trapezoidal rule and find
\begin{align} 
\int_{-\infty}^{\infty} x(t) e^{-2 \pi i f_k t} \ud t &\simeq \int_{0}^{T} x(t) e^{-2 \pi i f_k t} \ud t \nonumber \\
&\approx \sum_{m=1}^{M} x_m e^{-2 \pi i f_k t_m} \Delta t \nonumber \\
& = \sum_{m=1}^{M} x_m e^{-2 \pi i k m/M} \Delta t \equiv \widetilde{X}_k \Delta t M. \label{eqSA:dft1}
\end{align}
In the last line we have defined the discrete Fourier transform (DFT) of a time series $x_m$
\begin{equation} \label{eqSA:DFT}
\widetilde{X}_k (f_k) = \frac{1}{M}  \sum^M_{m=1} x_m e^{-2\pi i k m/M}.
\end{equation}

\section{The short-time Fourier transform}
The disadvantage of the Fourier transform is that it only gives you information
about the frequency spectrum of the full-time signal. If we are analysing a 
signal with a varying frequency structure it will be useful to extract time-frequency information about the signal. This is done by separating
the signal into shorter segments and then calculating the Fourier transform of each
segment. In this way, we obtain spectral information about the
signal at different times.

There are several ways to segment the signal, the method used in this thesis is to slide a time-window of length $\tau$ over the time signal, $x(t)$, in an iterative
process. In each iteration, the window is shifted forward in time by $\Delta \tau$.
This define a set of functions
\begin{equation}
S^i \equiv x(t) \big[ H(t+i \Delta \tau) - H(t + \tau + i \Delta \tau) \big] \qquad i = 0,1,2,3 ....,
\end{equation}
where $H(t)$ is the Heaviside step function.
The short-time Fourier transform (STFT) of $x(t)$ is then
\begin{equation}\label{eqSA:STFT}
\text{STFT}[x(t)] \equiv \int_{-\infty}^{\infty} S^i e^{-2 \pi i f t} \ud t 
= \widetilde{S^i}_k \qquad i = 0,1,2,3 ....,                                                         
\end{equation}

\section{The Nyquist frequency and aliasing}
An important question when sampling a continuous 
time signal is how may samples are required to accurately represent the
signal.
\begin{mdframed}[nobreak=true]
\paragraph{The Nyquist sampling theorem}
Let $x_b(t)$ be a band-limited signal, that is
\begin{equation}
\widetilde{x_b}(f) = 0 \quad \text{if} f > f_N,
\end{equation}
where $f_N$ is the band-limit of $x_b$. If $x_b$ is sampled with a sampling frequency
\begin{equation}
f_s \geq 2 f_N
\end{equation}
then the signal is uniquely determined by its samples. The frequency $f_N$ is referred to 
as the Nyquist frequency.
\end{mdframed}

If a signal is sampled at a lower rate than half the Nyquist frequency then
the signal will not be accurately represented and what is know as aliasing occurs.
Suppose that we sample a signal which a sampling rate $f_s$ and that the signal contains oscillations 
at frequencies larger than $f_s /2$. Then the 
DFT will not be able to distinguish between oscillations with 
frequencies $f_c > f_s/2$ and $f - 2 f_s$. This means that oscillations at a frequency
$f_c$ will be aliased down into to the frequency
\begin{equation}
f_a = f_c - f_N.
\end{equation}
In the case of hydrodynamic simulations, aliasing can appear when data is not saved
to disk frequently enough. In large 3D simulations, it is often unfeasible to save all the hydro-data for each individual time-step. It is, therefore, common to
save data to disk at a given time interval. The simulations which the GW signals presented
in this work are based on saved data roughly two times per millisecond of simulated time, which results in a Nyquist frequency of
1000 Hz. 

\section{The spectral energy density of a discrete time signal}
\eq{eqT:dedf} gives the expression for the spectral energy density of a continuous time signal,
it is obtained by directly applying Parseval's theorem to \eq{eqT:energy}.
\begin{mdframed}[nobreak=true]
\paragraph{Parseval's theorem}
If $\widetilde{x}$ is the Fourier transform of $x(t)$ then
\begin{equation}
\int_{-\infty}^{\infty} |x(t)|^2 \, \mathrm{d} t \, = \, \int_{-\infty}^{\infty} |\widetilde{x}(f)|^2 \, \mathrm{d} f. 
\end{equation}  
\end{mdframed}
To derive the corresponding expression for a discrete time series, first consider the quantity
\begin{equation}
\Gamma = \int_{-\infty}^{\infty} g(t)^2 \, \mathrm{d} t = \int_{-\infty}^{\infty} |\widetilde{g}|^2 \, \mathrm{d} f.  
\end{equation}
%\begin{equation} \label{eqSA:start}
%\Gamma = \int_{-\infty}^{\infty} |\widetilde{g}|^2 \, \mathrm{d} f.  
%\end{equation}
Next we construct the time-series $g_n$ by sampling $g(t)$ $M$ times over a time period $T$.
We then use the same procedure we used to derive the \eq{eqSA:dft1}, that is
to approximate the integral using the trapezoidal rule. This procedure yields
\begin{align} 
\Gamma &= \int_{-\infty}^{\infty} |\widetilde{g}|^2 \, \mathrm{d} f \nonumber \\ 
& \approx \sum_{k=1}^{M} \Big| \sum_{m=1}^{M} x_m e^{-2 \pi i k m/M} \Delta t \Big|^2 \Delta f \nonumber \\
& = \sum_{k=1}^{M} \big|\widetilde{g}_k \big|^2 T^2 \Delta f =  \sum_{k=1}^{M} \big|\widetilde{g}_k \big|^2 \, T,
\end{align}
where we in the last line have used that $\Delta f = 1/T$ and $\Delta t = T/M$ 
(the $1/M^2$ is absorbed in to the DFT). 
This implies that 
\begin{align}
\frac{\mathrm{d}}{\mathrm{d} f} \Gamma &= \frac{\mathrm{d}}{\mathrm{d} f} \int_{-\infty}^{\infty} |\widetilde{g}|^2 \mathrm{d} f \nonumber \\
&\approx \big|\widetilde{g}_k \big|^2 \, T^2.
\end{align}
The last step is to replace the placeholder function $g$ by $\dddot{Q}_{ij}$ and use the fact that
\begin{equation}
\widetilde{\dot{x(t)}} = 2\pi f \widetilde{x(t)}.  
\end{equation}
We then find that 
\begin{equation}
\frac{\mathrm{d}}{\mathrm{d} f} E \approx \Bigg[\frac{\Delta E}{\Delta f}\Bigg]_k = \big|\widetilde{\ddot{Q}^{ij}}_k \widetilde{\ddot{Q}^{ij}}_k| \, T^2.  
\end{equation}
Here we have moved the $ij$ subscript of $Q_{ij}$ up for better readability. 

% \section{Spectral leakage and windowing}
% If the time-signal only consists of oscillations at frequencies that are
% multiples of the sampling frequency we can accurately represent it with a DFT. However,
% if a signal contains frequency components that are not integer multiples of frequency 
% what is know as spectral leakage. The problem is what happens to the spectral energy 
% at a frequency that lies between two frequency bins, $f_k$ and $f_{k+1}$. The result is that
% this energy is ``leaks'' into other bins, the spectral energy can, in fact, be spread out
% over the whole range of frequency bins.  











