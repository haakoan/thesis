
%%% Local Variables:
%%% mode: latex
%%% TeX-master: "../../doktorarbeit"
%%% End:

\chapter{Theory}


\section{General relativity}
The total gravitational action is given by the sum of the 
Einstein action, $S_E$, and the matter action, $S_M$.
\begin{equation}
S_E = \frac{c^3}{16 \pi G} \int \mathrm{d}^4 x \sqrt{-g} R
\end{equation}


\section{Linearised theory}
One of the most straight forward ways to understand 
GWs is to expand the Einstein equations around the 
flat Minkowski space. Mathematically this means that we write the
metric tensor $\munu{g}$ as
\begin{equation} \label{eqT:metric}
\munu{g} = \munu{\eta} + \munu{h},
\end{equation}
where $\munu{\eta}$ is the Minkowski metric tensor and $\munu{h}$
is some small perturbation satisfying the condition
\begin{equation} \label{eqT:hsmall}
|\munu{h}|  \ll 1.
\end{equation}
The condition given by Eq.~\ref{eqT:hsmall} will not hold in a arbitrary 
reference frame. Therefor, by imposing the smallness condition on $\munu{h}$
we implicitly chose a frame where the numerical value of the components of $\munu{h}$ is much smaller than one,
in the region of space which we are interested in. In linearised theory we use the Minkowski metric tensor
to lower and raise indices. 

The field equations of general relativity can
be written in terms of the Ricci tensor ($\munu{R}$), the Ricci scalar ($R$), the metric tensor,
and the energy-momentum tensor ($\munu{T}$) as follows
\begin{equation} \label{eqT:einstein}
\munu{R} - \frac{1}{2} \munu{g} R = \frac{8 \pi G}{c^4 } \munu{T}. 
\end{equation}
Before we combine Eq.~\ref{eqT:metric} and Eq.~\ref{eqT:einstein} we introduce the simplifying notation
\begin{equation} \label{eqT:h}
h = \munut{\eta}\munu{h},
\end{equation}
and
\begin{equation} \label{eqT:hbar}
\munu{\bar{h}} = \munu{h} - \frac{1}{2}\munu{\eta}h.
\end{equation}
By inserting our expression for the metric tensor (Eq.~\ref{eqT:metric}) into Eq.~\ref{eqT:einstein}
and expand to linear order in $\munu{h}$ we find the linearised version of the Einstein equations
\begin{equation} \label{eqT:einlin}
\partial_{\gamma} \partial^{\gamma} \munu{\bar{h}} + \munu{\eta} \partial^{\rho} \partial^{sigma} \bar{h}_{\rho \sigma}
- \partial^{\rho}\partial_{\nu}\bar{h}_{\mu \sigma} -  \partial^{\rho}\partial_{\mu}\bar{h}_{\nu \sigma}
= \frac{8 \pi G}{c^4 } \munu{T}.
\end{equation}
We can simplify Eq.~\ref{eqT:einlin} by using the gauge freedom of our linearised theory to chose the Lorentz gauge,
\begin{equation} \label{eqT:lor}
\partial^{nu} \munu{\bar{h}} = 0.
\end{equation}
Under this gauge condition Eq.~\ref{eqT:einlin} reduces to a wave equation 
\begin{equation} \label{eqT:wave}
\partial_{\gamma} \partial^{\gamma} \munu{\bar{h}} = \frac{8 \pi G}{c^4 } \munu{T},
\end{equation}
since every term, except the first one, on the left hand side vanishes. 

Eq.~\ref{eqT:wave} further simplifies when we are outside of the sources generating
GWs, in vacuum the energy-momentum tensor vanishes and we get  
\begin{equation} \label{eqT:wavevacuum}
\partial_{\gamma} \partial^{\gamma} \munu{\bar{h}} = 0,
\end{equation}
which can be rewritten as
\begin{equation} \label{eqT:wavevacumm2}
\frac{1}{c^2} \partial_t^2 \munu{\bar{h}} = [ \partial_1^x  + \partial_y^2 + \partial_z^2] \munu{\bar{h}}.
\end{equation}
It is clear from the latter form that GWs propagate through spacetime at the speed of light in a wave-like fashion.

\section{The transverse-traceless gauge}
Even though we introduced the Lorentz gauge earlier, we have not completely removed all superfluous degrees of freedom 
in the linearised field equations. In vacuum, where the energy-momentum tensor vanishes and Eq.~\ref{eqT:wavevacumm} holds, it is possible
to simplify the expression for $\munu{h}$. The transverse-traceless gauge (we will denote the transverse-traceless gauge
with TT and quantities with a TT are understood to be in the TT-gauge) imposed the following conditions
\begin{equation} \label{eqT:ttg}
h^{0 \mu} = 0, \qquad h^{i}_{i} = 0, \qquad \text{and} \qquad \partial^j h_{ij} = 0.
\end{equation}
The solutions to Eq.~\ref{eqT:wavevacumm} are plane wave solutions and in the TT-gauge the 
solution for a plane wave propagating along the z-axis is given by
\begin{equation} \label{eqT:hzdir}
h_{ij}^{TT}=
  \begin{pmatrix}
    h_{+} & h_{\times} & 0  \\
    h_{\times} & -h_{+} & 0 \\
    0 & 0 & 0
  \end{pmatrix}_{ij}
  cos[\omega(t - z/c)].
\end{equation}
Here we $t$ denotes time, $\omega$ the angular frequency of the wave,
$h_+$ denotes the wave strain of the plus-polarised mode and $h_{\times}$ is the strain amplitude of the cross-polarised mode.
To prove that we can impose the conditions of Eq.~\ref{eqT:ttg} we start by
realising that the Lorentz gauge does not completely constrain the theory by considering the 
coordinate transformation 
\begin{equation} \label{eqT:ct}
x^{\mu} \rightarrow x^{'\mu} = x^{\mu} + \epsilon^{\mu},
\end{equation}
where $\epsilon^{\mu}$ satisfy $\partial_{\gamma} \partial^{\gamma} \epsilon_{\mu} = 0$ 
and $|\partial_{\nu} \epsilon_{\mu}|$ is at the most on the order of $|\munu{h}|$.
Under an arbitrary coordinate transformation $x^{\mu} \rightarrow x^{'\mu}(x)$ the second rank tensor $\munu{h}$ transforms as
\begin{equation} \label{eqT:t2c}
\munu{h} \rightarrow \munu{h'} = \frac{\partial x^{\gamma}}{\partial x^{'\mu}} \frac{\partial x^{\sigma}}{\partial x^{'\nu}} h_{\gamma \sigma}.
\end{equation}
Evaluating Eq.~\ref{eqT:t2c} for the coordinate transformation given by Eq.~\ref{eqT:ct} gives
\begin{equation} \label{eqT:htrans}
\munu{h} \rightarrow \munu{h'} = \munu{h} - (\partial_{\mu} \epsilon_{\nu} + \partial_{\nu} \epsilon_{\mu}). 
\end{equation}
By combining Eq.~\ref{eqT:htrans} and Eq.~\ref{eqT:hbar} we find that under Eq.~\ref{eqT:ct} $\munu{\bar{h}}$ transforms as follows
\begin{equation} \label{eqT:hbartrans}
\munu{\bar{h}} \rightarrow \munu{\bar{h}'} =  \munu{{h}} - (\partial_{\mu} \epsilon_{\nu} + \partial_{\nu} \epsilon_{\mu} - \munu{\eta}\partial_{\gamma} \epsilon^{\gamma}). 
\end{equation}
By applying  $\partial_{\mu}$ to Eq.~\ref{eqT:ct} we find
\begin{equation} \label{eqT:dert}
\partial_{\mu} x^{'\mu} = \partial_{\mu} x^{\mu} + \partial_{\mu} \epsilon^{\mu},
\end{equation}
since $|\partial_{\mu} \epsilon^{\mu}| \ll 1$ Eq.~\ref{eqT:dert} implies that
\begin{equation}
\partial_{\mu} x^{'\mu} = \frac{\partial x^{'\mu} }{\partial x^{\mu}} = 1.
\end{equation}
This means that under the transformation given by Eq.~\ref{eqT:ct} the derivatives
transforms as follows
\begin{equation} \label{eqT:dert2}
\partial_{\mu} \rightarrow \partial_{\mu}^{'} = \partial_{\mu}.
\end{equation}
We can now calculate how Eq.~\ref{eqT:lor} transform under Eq.~\ref{eqT:ct}
\begin{align} \label{eqT:lort}
\partial^{\nu} \munu{\bar{h}} & \rightarrow (\partial^{\nu} \munu{\bar{h}})^{'} \\ \nonumber
&= \partial^{\nu} \left [ \munu{\bar{h}} - (\partial_{\mu} \epsilon_{\nu} + \partial_{\nu} \epsilon_{\mu} - \munu{\eta}\partial_{\gamma} \epsilon^{\gamma}) \right ] \\
&= \partial^{\nu} \munu{\bar{h}} - \partial^{\gamma} \partial_{\gamma} \epsilon_{\mu} = 0. \nonumber
\end{align}
We can now directly see that the transformation does not break the Lorentz gauge condition.

Instead of thinking about how $\munu{\bar{h}}$ transform under Eq.~\ref{eqT:ct}
we can instead think about it like this: Eq.~\ref{eqT:hbartrans} tells us that
that from four independent functions $\epsilon_{mu}$ we can construct the functions 
\begin{equation}
\munu{\epsilon} \equiv \partial_{\mu} \epsilon_{\nu} + \partial_{\nu} \epsilon_{\mu} - \munu{\eta}\partial_{\gamma} \epsilon^{\gamma}
\end{equation}
and are free to subtract this functions from $\munu{\bar{h}}$ without breaking the gauge condition set by Eq.~\ref{eqT:lor}.
With this freedom we can now choose the four functions such that they impose four simplifying conditions on  $\munu{h}$.
In the TT-gauge the four functions are chosen in such a that the trace of $\munu{\bar{h}}$ is zero and
so that $h^{0i} = 0$. Note that if the trace of $\munu{\bar{h}}$ vanishes then $\munu{\bar{h}} = \munu{h}$.
These four conditions, together with the Lorentz gauge, is what defines the TT-gauge and results in the
conditions set by Eq.~\ref{eqT:ttg}.

\section{Generation of gravitational waves}
We turn now to the to the generation of gravitational waves, in the linearised theory framework,
we start by writing down of Eq.~\ref{eqT:einlin} for a generic source under the assumption 
that the gravitational field generated by the source is weak enough to justify the expantion
around flat spacetime. 

As for any wave equation, the solution of \eq{eqT:einstein} can be found by integrating over the source
\begin{equation} \label{eqT:sol}
\munu{\bar{h}}(t,\matbf{x}) = \frac{4G}{c^4} \int \mathrm{d}^3 x' \frac{\munu{T}(t-|\mathbf{x}-\mathbf{x'}|/c}{|\mathbf{x}-\mathbf{x'}|}.
\end{equation}
Far way from the source at a distance $D$, if the velocities ($v$) within the source is small compared to the speed of light, \eq{eqT:sol} reduces to the Einsteins famous quadrupole expression. In the TT-gauge the quadrupole expression can be written as follows  
\begin{equation} \label{eqT:equad}
\munu{h}(t,\matbf{x}) = \frac{1}{D} \frac{4G}{c^4} \left [P_{im} P_{jn} - \frac{1}{2} P_{ij} P_{mn} \right ] \ddot{Q}_{ij} (t -d/c),
\end{equation}
where $Q_{ij}$ is the the mass quadrupole moment that to leading order in $v/c$ is given by
\begin{equation} \label{eqT:quad}
Q_{ij} = \int \mathrm{d}^3 x \rho(t,\matbf{x})\left [x_{ij} - \frac{1}{3} \delta^{ij} x_l x_l \right ].
\end{equation}
$P_{ij} = \delta_{ij} -n_i n_j$ is the projection operator onto the plane transverse to the direction the wave is propagating in,
$\hat{n}_i = x_i / \sqrt{x_j _x^j}$. To find the equation describing $h_{+}$ and $h_{\times}$ for a wave propagating in the general direction
$\hat{n}$ we first consider a wave propagating along the z-axis in a coordinate system with axes $(x,y,z)$. 
If $\hat{n} = \hat{z}$ then $P_{ij}$ becomes and we find that 
\begin{equation} \label{eqT:pjz}
P_{im} P_{jn} - \frac{1}{2} P_{ij} P_{mn} \right ] \ddot{Q}_{ij} = 
  \begin{pmatrix}
    (\ddot{Q}_{11}-\ddot{Q}_{22})/2 & \ddot{Q}_{12} & 0  \\
    \ddot{Q}_{21} & (\ddot{Q}_{22} - \ddot{Q}_{11})/2 & 0 \\
    0 & 0 & 0
  \end{pmatrix}_{ij}.
\end{equation}
By comparing \eq{eqT:hzdir} and \eq{eqT:pjz} we see that
\begin{align} \label{eqT:zhchx}
h_{+}^{TT} &= \frac{G}{c^4 D} (\ddot{Q}_{11}-\ddot{Q}_{22}) \\ \nonumber
h_{\times}^{TT} &= frac{2G}{c^4 D} \ddot{Q}_{12}.
\end{align}
Now consider a wave propegating along the $\hat{n} = (\sin{\theta} \sin{\phi}, \sin{\theta} \cos{\phi},\cos{\theta})$.
We can view this as a wave propagating along the z'-axis of a coordinate system, with axes $(x',y',z')$, that has been constructed  
by rotating the original system about the z-axis by an angle $\phi$ and then about the x-axis by an angle $\theta$. $h_{+}$ and $h_{\times}$.
The rotation matrix R of the two consecutive rotations is
\begin{equation} \label{eqT:pjz}
R = 
  \begin{pmatrix}
    \cos{\phi} & \sin{\phi} & 0  \\
    -\sin{\phi} & \cos{\phi} & 0 \\
    0 & 0 & 0
  \end{pmatrix}
  \begin{pmatrix}
    0 & 0 & 0  \\
    0& \cos{\theta} & \sin{\theta} \\
    0 & \cos{\theta} & \cos{\theta}
  \end{pmatrix}
\end{equation}

Since the wave propegates along the z'-direction we can use the result from before (\eq{eqT:zhchx})
\begin{align} \label{eqT:zhchxp}
h_{+}^{TT} &= \frac{G}{c^4 D} (\ddot{Q}_{11}'-\ddot{Q}_{22}') \\ \nonumber
h_{\times}^{TT} &= frac{2G}{c^4 D} \ddot{Q}_{12}'.
\end{align}
A second rank tensor, like the quadrupole tensor, transforms under as follows

\begin{eqnarray}
\label{eq:qtp}
\ddot{Q}_{\theta \phi} =&  \left (\ddot{Q}_{22} - \ddot{Q}_{11} \right ) \cos{\theta}\sin{\phi}\cos{\phi} \nonumber \\
&+ \ddot{Q}_{12} \cos{\theta} \left (\cos^2 \phi - \sin^2 \phi \right ) \nonumber \\ 
&+ \ddot{Q}_{13} \sin \theta \sin \phi - \ddot{Q}_{23} \sin \theta \cos\phi,
\end{eqnarray}
\begin{eqnarray}
\ddot{Q}_{\phi \phi} &= \ddot{Q}_{11} \sin^2 \phi + \ddot{Q}_{22} \cos^2 \phi - 2 \ddot{Q}_{12} \sin{\phi}\cos{\phi}
\end{eqnarray}
and
\begin{eqnarray}
\ddot{Q}_{\theta \theta} &= \left ( \ddot{Q}_{11} \cos^2 \phi + \ddot{Q}_{22} \sin^2 \phi +  2 \ddot{Q}_{12} \sin{\phi} \cos{\phi} \right) \cos^2 \theta \nonumber \\
&+ \ddot{Q}_{33} \sin^2 \theta - 2 \left (\ddot{Q}_{13} \cos{\phi} + \ddot{Q}_{23} \sin{\phi} \right ) \sin{\theta} \cos{\theta}. 
\end{eqnarray}


In the transverse traceless (TT) gauge and the far-field limit the metric perturbation, $\mathbf{h^\mathrm{TT}}$, 
can be expressed in terms of the amplitudes of the two independent polarisation modes in the following way,
\begin{equation}
\mathbf{h}^\mathrm{TT}(\mathbf{X},t) = \frac{1}{D}   \left [ A_{+} \mathbf{e}_{+} + A_{\times} \mathbf{e}_{\times} \right ].
\end{equation}
Here, $D$ denotes the distance between the source and the observer, $A_+$ denotes the wave amplitude of the plus-polarised mode, $A_\times$ is the wave amplitude of the cross-polarised mode and
$\mathbf{e}_{\times}$ and $\mathbf{e}_{+}$ denote the unit polarisation tensors. The unit polarisation tensors are given by
\begin{equation}
\mathbf{e}_{+}  = \mathbf{e}_{\theta} \otimes \mathbf{e}_{\theta} - \mathbf{e}_{\phi} \otimes \mathbf{e}_{\phi},
\end{equation}
\begin{equation}
\mathbf{e_{\times}} = \mathbf{e}_{\theta} \otimes \mathbf{e}_{\phi} + \mathbf{e}_{\phi} \otimes \mathbf{e}_{\theta},
\end{equation}
where $\mathbf{e}_{\theta}$ and $\mathbf{e}_{\phi}$ are the unit vectors in the $\theta$ and $\phi$
direction of a spherical coordinate system and $\otimes$ denotes the tensor product.
Using the quadrupole approximation in the slow-motion limit, the amplitudes $A_{\times}$ and $A_{+}$ can be computed from the second time derivative 
of the symmetric trace-free (STF) part of the mass quadrupole tensor $Q$~\citep{oohara_97},
\begin{equation}
\label{eq:aplus}
A_{+} = \ddot{Q}_{\theta \theta} - \ddot{Q}_{\phi \phi},
\end{equation}
\begin{equation}
\label{eq:ax}
A_{\times} =2 \ddot{Q}_{\theta \phi}.
\end{equation}
The components of $Q$ in the orthonormal basis associated with
spherical polar coordinates used in this formula can be obtained from
the Cartesian components $\ddot{Q}_{ij}$ of $\ddot{Q}$ \citep{oohara_97,nakamura_87}.  Using the
continuity and momentum equations to eliminate time derivatives
\citep{oohara_97,finn_89,blanchet_90}, the Cartesian components can be obtained as:
\begin{equation} \label{eq:STFQ}
\ddot{Q}_{ij} =\mathrm{STF} \left [2 \frac{G}{c^4} \int \ud^3 x \, \rho \left ( v_i v_j - x_i \partial_j \Phi \right ) \right].
\end{equation}
Here, $G$ is Newton's gravitational constant, $c$ is the speed of
light, and $v_i$ and $x_i$ are the Cartesian velocity components and
coordinates ($i = 1,2,3$), respectively. The gravitational potential
$\Phi$ is the gravitational potential used in the simulations (with post-Newtonian
corrections taken into account). $\mathrm{STF}$ denotes the projection operator
onto the symmetric trace-free part.
The advantage of this form is that
the second-order time derivatives are transformed into first-order
spatial derivatives, thus circumventing problems associated with the
numerical evaluation of second-order time derivatives. 
Using standard
coordinate transformations between Cartesian and spherical
coordinates, we obtain \citep{oohara_97,nakamura_87} the components 
$\ddot{Q}_{\theta \theta}$, $\ddot{Q}_{\phi \phi}$, and
$\ddot{Q}_{\theta \phi}$ needed in Eq.~(\ref{eq:aplus}) and (\ref{eq:ax}),
\begin{eqnarray}
\label{eq:qtp}
\ddot{Q}_{\theta \phi} =&  \left (\ddot{Q}_{22} - \ddot{Q}_{11} \right ) \cos{\theta}\sin{\phi}\cos{\phi} \nonumber \\
&+ \ddot{Q}_{12} \cos{\theta} \left (\cos^2 \phi - \sin^2 \phi \right ) \nonumber \\ 
&+ \ddot{Q}_{13} \sin \theta \sin \phi - \ddot{Q}_{23} \sin \theta \cos\phi,
\end{eqnarray}
\begin{eqnarray}
\ddot{Q}_{\phi \phi} &= \ddot{Q}_{11} \sin^2 \phi + \ddot{Q}_{22} \cos^2 \phi - 2 \ddot{Q}_{12} \sin{\phi}\cos{\phi}
\end{eqnarray}
and
\begin{eqnarray}
\ddot{Q}_{\theta \theta} &= \left ( \ddot{Q}_{11} \cos^2 \phi + \ddot{Q}_{22} \sin^2 \phi +  2 \ddot{Q}_{12} \sin{\phi} \cos{\phi} \right) \cos^2 \theta \nonumber \\
&+ \ddot{Q}_{33} \sin^2 \theta - 2 \left (\ddot{Q}_{13} \cos{\phi} + \ddot{Q}_{23} \sin{\phi} \right ) \sin{\theta} \cos{\theta}. 
\end{eqnarray}

In axisymmetry the only independent component of $\mathbf{h}^\mathrm{TT}$ is 
\begin{equation}
\mathbf{h}^\mathrm{TT}_{\theta \theta} = \frac{1}{8}\sqrt{\frac{15}{\pi}} \sin^2{\theta} \frac{A_{20}^\mathrm{E2}}{D},
\end{equation}
where $D$ is the distance to the source, $\theta$ is the inclination angle of the observer with respect to the
axis of symmetry, and $A_{20}^\mathrm{E2}$ represents the only non-zero quadrupole amplitude.
In spherical coordinates $A_{20}^\mathrm{E2}$ can be expressed as follows
\begin{eqnarray} \label{eq:2dquad}
A_{20}^\mathrm{E2} (t) =  \frac{G}{c^4} \frac{16 \pi^{3/2}}{\sqrt{15}} \int_{-1}^{1}\int^{\infty}_0 \rho \left [ v_r^2(3 z^2 - 1)+ \right. \nonumber \\
v_{\theta}^2(2-3 z^2) - v_{\phi}^2 - 6 v_r v_{\theta} z\sqrt{1-z^2} + \nonumber \\
r \partial_r \Phi (3 z^2 - 1) +\left. 3 \partial_{\theta}z\sqrt{1-z^2} \Phi \right ]r^2 dr \, dz.
\end{eqnarray}
Here, $v_i$ and $\partial_i$ ($i = r, \theta, \phi$) represent the velocity components and derivatives, respectively, along
the basis vectors of the spherical coordinate system, and $z \equiv \cos \theta$.
For details we refer the reader to \cite{mueller_97}.
