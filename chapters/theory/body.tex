
%%% Local Variables:
%%% mode: latex
%%% TeX-master: "../../doktorarbeit"
%%% End:

\chapter{Theory}


\section{General relativity}
The total gravitational action is given by the sum of the 
Einstein action, $S_E$, and the matter action, $S_M$.
\begin{equation}
S_E = \frac{c^3}{16 \pi G} \int \mathrm{d}^4 x \sqrt{-g} R
\end{equation}


\section{Linearised theory}
One of the most straight forward ways to understand 
GWs is to expand the Einstein equations around the 
flat Minkowski space. Mathematically this means that we write the
metric tensor $\munu{g}$ as
\begin{equation} \label{eqT:metric}
\munu{g} = \munu{\eta} + \munu{h},
\end{equation}
where $\munu{\eta}$ is the Minkowski metric tensor and $\munu{h}$
is some small perturbation satisfying the condition
\begin{equation} \label{eqT:hsmall}
|\munu{h}|  \ll 1.
\end{equation}
The condition given by Eq.~\ref{eqT:hsmall} will not hold in a arbitrary 
reference frame. Therefor, by imposing the smallness condition on $\munu{h}$
we implicitly chose a frame where the numerical value of the components of $\munu{h}$ is much smaller than one,
in the region of space which we are interested in. In linearised theory we use the Minkowski metric tensor
to lower and raise indices. 

The field equations of general relativity can
be written in terms of the Ricci tensor ($\munu{R}$), the Ricci scalar ($R$), the metric tensor,
and the energy-momentum tensor ($\munu{T}$) as follows
\begin{equation} \label{eqT:einstein}
\munu{R} - \frac{1}{2} \munu{g} R = \frac{8 \pi G}{c^4 } \munu{T}. 
\end{equation}
Before we combine Eq.~\ref{eqT:metric} and Eq.~\ref{eqT:einstein} we introduce the simplifying notation
\begin{equation} \label{eqT:h}
h = \munut{\eta}\munu{h},
\end{equation}
and
\begin{equation} \label{eqT:hbar}
\munu{\bar{h}} = \munu{h} - \frac{1}{2}\munu{\eta}h.
\end{equation}
By inserting our expression for the metric tensor (Eq.~\ref{eqT:metric}) into Eq.~\ref{eqT:einstein}
and expand to linear order in $\munu{h}$ we find the linearised version of the Einstein equations
\begin{equation} \label{eqT:einlin}
\partial_{\gamma} \partial^{\gamma} \munu{\bar{h}} + \munu{\eta} \partial^{\rho} \partial^{sigma} \bar{h}_{\rho \sigma}
- \partial^{\rho}\partial_{\nu}\bar{h}_{\mu \sigma} -  \partial^{\rho}\partial_{\mu}\bar{h}_{\nu \sigma}
= \frac{8 \pi G}{c^4 } \munu{T}.
\end{equation}
We can simplify Eq.~\ref{eqT:einlin} by using the gauge freedom of our linearised theory to chose the Lorentz gauge,
\begin{equation} \label{eqT:lor}
\partial^{nu} \munu{\bar{h}} = 0.
\end{equation}
Under this gauge condition Eq.~\ref{eqT:einlin} reduces to a wave equation 
\begin{equation} \label{eqT:wave}
\partial_{\gamma} \partial^{\gamma} \munu{\bar{h}} = \frac{8 \pi G}{c^4 } \munu{T},
\end{equation}
since every term, except the first one, on the left hand side vanishes. 

Eq.~\ref{eqT:wave} further simplifies when we are outside of the sources generating
GWs, in vacuum the energy-momentum tensor vanishes and we get  
\begin{equation} \label{eqT:wavevacuum}
\partial_{\gamma} \partial^{\gamma} \munu{\bar{h}} = 0,
\end{equation}
which can be rewritten as
\begin{equation} \label{eqT:wavevacumm2}
\frac{1}{c^2} \partial_t^2 \munu{\bar{h}} = [ \partial_1^x  + \partial_y^2 + \partial_z^2] \munu{\bar{h}}.
\end{equation}
It is clear from the latter form that GWs propagate through spacetime at the speed of light in a wave-like fashion.

\section{The transverse-traceless gauge}
Even though we introduced the Lorentz gauge earlier, we have not completely removed all superfluous degrees of freedom 
in the linearised field equations. The transverse-traceless gauge (we will denote quantities in
the transverse-traceless gauge with by a TT superscript) imposed the following conditons
\begin{equation}
h^{0 \mu} = 0, \qquad h^{i}_{i} = 0, \qquad \text{and} \qquad \partial^j h_{ij} = 0
\end{equation}
To prove that we are free to impose these conditions we observe that a coordinate transformation 
\begin{equation} \label{eqT:ct}
x^{\mu} \rightarrow x^{'\mu} = x^{\mu} + \epsilon^{\mu},
\end{equation}
where $\epsilon^{\mu}$ satisfy $\partial_{\gamma} \partial^{\gamma} \epsilon_{\mu} = 0$ 
and $|\partial_{\nu} \epsilon_{\mu}|$ is at the most on the order of $|\munu{h}|$,
does not break the condition imposed by the Lorentz gauge (Eq.~\ref{eqT:lor}).
Under an arbitrary coordinate transformation $x^{\mu} \rightarrow x^{'\mu}(x)$ the second rank tensor $\munu{h}$ transforms as
\begin{equation} \label{eqT:t2c}
\munu{h} \rightarrow \munu{h'} = \frac{\partial x^{\gamma}}{\partial x^{'\mu}} \frac{\partial x^{\sigma}}{\partial x^{'\nu}} h_{\gamma \sigma}.
\end{equation}
Under 

Evaluating Eq.~\ref{eqT:t2c} for the coordinate transformation given by Eq.~\ref{eqT:ct} gives
\begin{equation} \label{eqT:htrans}
\munu{h} \rightarrow \munu{h'} = \munu{h} - (\partial_{\mu} \epsilon_{\nu} + \partial_{\nu} \epsilon_{\mu}). 
\end{equation}
By combining Eq.~\ref{eqT:htrans} and Eq.~\ref{eqT:hbar} we find that under Eq.~\ref{eqT:ct} $\munu{\bar{h}}$ transforms as follows
\begin{equation} \label{eqT:hbartrans}
\munu{\bar{h}} \rightarrow \munu{\bar{h}'} =  \munu{{h}} - (\partial_{\mu} \epsilon_{\nu} + \partial_{\nu} \epsilon_{\mu} - \munu{\eta}\partial_{\gamma} \epsilon^{\gamma}). 
\end{equation}
We can now directly see that the Lorentz gauge condition is satisfied, since $(\partial_{\mu} \partial^{\mu})\partial_{\nu} = \partial_{\nu}(\partial_{\mu} \partial^{\mu})$ (the derivatives commute). This means that from the four independent functions $\epsilon_{mu}$ we can construct the functions 
\begin{equation}
\munu{\epsilon} \equiv \partial_{\mu} \epsilon_{\nu} + \partial_{\nu} \epsilon_{\mu} - \munu{\eta}\partial_{\gamma} \epsilon^{\gamma}
\end{equation}
and are free to subtract this functions from $\munu{\bar{h}}$ without breaking the gauge condition set by Eq.~\ref{eqT:lor}. 
With this freedom we can now choose the four functions $\epsilon^{\mu}$ such that they impose four simplifying conditions  $\munu{h}$.
In the transverse-traceless gauge the four functions are chosen in such a way that $\munut{\eta}\munu{\bar{h}} = 0$ and $\bar{h}^{0 i} = 0$.
Note that if the trace of $\munu{\bar{h}}$ is zero then $\munu{\bar{h}} = \munu{h}$. Under these conditions the Lorentz gauge condition
reduces to $\partial^j h_{ij} = 0$.
