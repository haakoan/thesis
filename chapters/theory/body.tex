
%%% Local Variables:
%%% mode: latex
%%% TeX-master: "../../doktorarbeit"
%%% End:

\chapter{Theory} \label{ch:theory}
% \section{General relativity}
% The total gravitational action is given by the sum of the 
% Einstein action, $S_E$, and the matter action, $S_M$.
% \begin{equation}
% S_E = \frac{c^3}{16 \pi G} \int \mathrm{d}^4 x \sqrt{-g} R
% \end{equation}
\section{Linearised theory}
One of the most straightforward ways to understand 
GWs is to expand the Einstein equations around 
Minkowski space. Mathematically this means that we write the
metric tensor, $\munu{g}$, as
\begin{equation} \label{eqT:metric}
\munu{g} = \munu{\eta} + \munu{h},
\end{equation}
where $\munu{\eta}$ is the Minkowski metric tensor and $\munu{h}$
is some small perturbation satisfying
\begin{equation} \label{eqT:hsmall}
|\munu{h}|  \ll 1.
\end{equation}
The condition given by Eq.~\ref{eqT:hsmall} will not hold in a arbitrary 
reference frame. Therefore, by imposing the smallness condition on $\munu{h}$
we implicitly chose a frame where the numerical value of the components of $\munu{h}$ is much smaller than one,
in the region of space which we are interested in. In linearised theory we use the Minkowski metric tensor
to lower and raise indices. 

The field equations of general relativity can
be written in terms of the Ricci tensor, $\munu{R}$, the Ricci scalar, $R$, the metric tensor,
and the energy-momentum tensor, $\munu{T}$, as follows
\begin{equation} \label{eqT:einstein}
\munu{R} - \frac{1}{2} \munu{g} R = \frac{8 \pi G}{c^4 } \munu{T}. 
\end{equation}
Before linerisaing the Einstein equations, by combining Eq.~\ref{eqT:metric} and Eq.~\ref{eqT:einstein}, we introduce some simplifying notation.
Introducing the quantity
\begin{equation} \label{eqT:h}
h = \munut{\eta}\munu{h},
\end{equation}
and
\begin{equation} \label{eqT:hbar}
\munu{\bar{h}} = \munu{h} - \frac{1}{2}\munu{\eta}h,
\end{equation}
will allow us to write the equations in a more compact form that is easier to work with.
By inserting the expression for the metric tensor (Eq.~\ref{eqT:metric}) into Eq.~\ref{eqT:einstein}
and expand to linear order in $\munu{h}$ we find the linearised version of the Einstein equations
\begin{equation} \label{eqT:einlin}
\partial_{\gamma} \partial^{\gamma} \munu{\bar{h}} + \munu{\eta} \partial^{\rho} \partial^{\sigma} \bar{h}_{\rho \sigma}
- \partial^{\sigma}\partial_{\nu}\bar{h}_{\mu \sigma} -  \partial^{\sigma}\partial_{\mu}\bar{h}_{\nu \sigma}
= -\frac{16 \pi G}{c^4 } \munu{T}.
\end{equation}
We can simplify Eq.~\ref{eqT:einlin} by using the gauge freedom of linearised theory to impose the Lorentz gauge
\begin{equation} \label{eqT:lor}
\partial^{\nu} \munu{\bar{h}} = 0.
\end{equation}
Under this gauge condition Eq.~\ref{eqT:einlin} reduces to a wave equation 
\begin{equation} \label{eqT:wave}
\partial_{\gamma} \partial^{\gamma} \munu{\bar{h}} = -\frac{16 \pi G}{c^4 } \munu{T},
\end{equation}
since every term, except the first one, on the left hand side vanishes. 

Eq.~\ref{eqT:wave} further simplifies when we are outside of the sources generating
GWs, in vacuum the energy-momentum tensor is zero and we get  
\begin{equation} \label{eqT:wavevacumm}
\partial_{\gamma} \partial^{\gamma} \munu{\bar{h}} = 0,
\end{equation}
which can be rewritten as
\begin{equation} \label{eqT:wavevacumm2}
\frac{1}{c^2} \partial_t^2 \munu{\bar{h}} = [ \partial^2_x  + \partial_y^2 + \partial_z^2] \munu{\bar{h}}.
\end{equation}
If we compare this expression for $\munu{\bar{h}}$ to a traditional wave equation, for example that of a sound wave
propagating though a fluid or a electromagnetic wave through vacuum, is becomes clear that GWs propagate through 
spacetime at the speed of light in a wave-like fashion.

\section{The transverse-traceless gauge}
Even though we introduced the Lorentz gauge above, we have not completely removed all non-physical degrees of freedom 
in the linearised field equations. In vacuum, where the energy-momentum tensor vanishes and Eq.~\ref{eqT:wavevacumm} holds, it is possible
to simplify the expression for $\munu{h}$. The transverse-traceless gauge (we will denote the transverse-traceless gauge
with TT and quantities with a TT are understood to be in the TT-gauge) imposed the following conditions
\begin{equation} \label{eqT:ttg}
h^{0 \mu} = 0, \qquad h^{i}_{i} = 0, \qquad \text{and} \qquad \partial^j h_{ij} = 0.
\end{equation}
The solutions to Eq.~\ref{eqT:wavevacumm} are plane wave solutions and in the TT-gauge the 
solution for a plane wave propagating along the z-axis is given by
\begin{equation} \label{eqT:hzdir}
h_{ij}^{TT}=
  \begin{pmatrix}
    h_{+} & h_{\times} & 0  \\
    h_{\times} & -h_{+} & 0 \\
    0 & 0 & 0
  \end{pmatrix}_{ij}
  \cos[\omega(t - z/c)].
\end{equation}
Here $t$ denotes time, $\omega$ the angular frequency of the wave,
$h_+$ denotes the strain of the plus-polarised mode and $h_{\times}$ is the strain amplitude of the cross-polarised mode.
To prove that we can impose Eq.~\ref{eqT:ttg} we start by
realising that the Lorentz gauge does not completely remove all the superfluous degrees of freedom in the theory. Consider the 
coordinate transformation 
\begin{equation} \label{eqT:ct}
x^{\mu} \rightarrow x^{'\mu} = x^{\mu} + \epsilon^{\mu},
\end{equation}
where $\epsilon^{\mu}$ satisfies $\partial_{\gamma} \partial^{\gamma} \epsilon^{\mu} = 0$, 
and $|\partial_{\nu} \epsilon_{\mu}|$ is at the most on the order of smallness as $|\munu{h}|$.
Under an arbitrary coordinate transformation $x^{\mu} \rightarrow x^{'\mu}(x)$ the second rank tensor $\munu{h}$ transforms as
\begin{equation} \label{eqT:t2c}
\munu{h} \rightarrow \munu{h'} = \frac{\partial x^{\gamma}}{\partial x^{'\mu}} \frac{\partial x^{\sigma}}{\partial x^{'\nu}} h_{\gamma \sigma}.
\end{equation}
Evaluating Eq.~\ref{eqT:t2c} for the coordinate transformation given by Eq.~\ref{eqT:ct} gives
\begin{equation} \label{eqT:htrans}
\munu{h} \rightarrow \munu{h'} = \munu{h} - (\partial_{\mu} \epsilon_{\nu} + \partial_{\nu} \epsilon_{\mu}). 
\end{equation}
By combining Eq.~\ref{eqT:htrans} and Eq.~\ref{eqT:hbar} we find that under Eq.~\ref{eqT:ct} $\munu{\bar{h}}$ transforms as
\begin{equation} \label{eqT:hbartrans}
\munu{\bar{h}} \rightarrow \munu{\bar{h}'} =  \munu{{h}} - (\partial_{\mu} \epsilon_{\nu} + \partial_{\nu} \epsilon_{\mu} - \munu{\eta}\partial_{\gamma} \epsilon^{\gamma}). 
\end{equation}
By applying  $\partial_{\mu}$ to Eq.~\ref{eqT:ct} we find
\begin{equation} \label{eqT:dert}
\partial_{\mu} x^{'\mu} = \partial_{\mu} x^{\mu} + \partial_{\mu} \epsilon^{\mu}.
\end{equation}
Since we have required that $|\partial_{\mu} \epsilon^{\mu}| \ll 1$,  Eq.~\ref{eqT:dert} implies that
\begin{equation}
\partial_{\mu} x^{'\mu} = \frac{\partial x^{'\mu} }{\partial x^{\mu}} = 1.
\end{equation}
This means that under the transformation given by Eq.~\ref{eqT:ct} the derivatives
transform as
\begin{equation} \label{eqT:dert2}
\partial_{\mu} \rightarrow \partial_{\mu}^{'} = \partial_{\mu}.
\end{equation}
We can now calculate how the Lorentz gauge conditions (Eq.~\ref{eqT:lor}) transform under Eq.~\ref{eqT:ct} and we
find:
\begin{align} \label{eqT:lort}
\partial^{\nu} \munu{\bar{h}} & \rightarrow (\partial^{\nu} \munu{\bar{h}})^{'} \\ \nonumber
&= \partial^{\nu} \left [ \munu{\bar{h}} - (\partial_{\mu} \epsilon_{\nu} + \partial_{\nu} \epsilon_{\mu} - \munu{\eta}\partial_{\gamma} \epsilon^{\gamma}) \right ] \\
&= \partial^{\nu} \munu{\bar{h}} - \partial^{\gamma} \partial_{\gamma} \epsilon_{\mu} = 0. \nonumber
\end{align}
Because we demanded that $\partial_{\gamma} \partial^{\gamma} \epsilon^{\mu} = 0$, we can now directly see 
that the transformation does not break the Lorentz gauge condition. In other words, we
are free to perform the coordinate transformation given by \eq{eqT:ct}.

Instead of thinking about how $\munu{\bar{h}}$ transforms under Eq.~\ref{eqT:ct}
we can also construct the functions 
\begin{equation}
\munu{\epsilon} \equiv \partial_{\mu} \epsilon_{\nu} + \partial_{\nu} \epsilon_{\mu} - \munu{\eta}\partial_{\gamma} \epsilon^{\gamma}
\end{equation}
from our four independent functions $\epsilon_mu$ and are free to subtract these functions from $\munu{\bar{h}}$ without breaking the gauge condition set by Eq.~\ref{eqT:lor}.
The tool we use to impose these four conditions is the coordinate transform given by \eq{eqT:ct}, with the constraints
on $\epsilon_{mu}$ and its derivatives specified above.
With this freedom, we can now choose the four functions such that they impose four simplifying conditions on  $\munu{h}$.
In the TT-gauge the four functions are chosen in such a way that the trace of $\munu{\bar{h}}$ is zero and
 $h^{0i} = 0$. Note that if the trace of $\munu{\bar{h}}$ vanishes $\munu{\bar{h}} = \munu{h}$, and
we will usually write $\munu{h}$ instead of $\munu{\bar{h}}$ when we are in the TT-gauge.
These four conditions, together with the Lorentz gauge, define the TT-gauge and result in the
conditions given by Eq.~\ref{eqT:ttg}.

\section{Generation of gravitational waves}
We turn now to the generation of gravitational waves. In the linearised theory framework,
we start by writing down the solution of Eq.~\ref{eqT:wave} for a generic source under the assumption that the gravitational field generated by the source is weak enough to justify the expansion around flat spacetime. 

As for any wave equation, the solution of \eq{eqT:wave} can be found by integrating over the source
\begin{equation} \label{eqT:sol}
\munu{\bar{h}}(t,\mathbf{x}) = \frac{4G}{c^4} \int \mathrm{d}^3 x' \frac{\munu{T}(t-|\mathbf{x}-\mathbf{x'}|/c}{|\mathbf{x}-\mathbf{x'}|}.
\end{equation}
Far way from the source at a distance $D$, if the velocities, $v$, within the source are small compared to the speed of light, \eq{eqT:sol} reduces to the famous Einstein quadrupole formula. In the TT-gauge the quadrupole formula can be written as follows  
\begin{equation} \label{eqT:equad}
\munu{h}(t,\mathbf{x}) = \frac{1}{D} \frac{4G}{c^4} \left [P_{im} P_{jn} - \frac{1}{2} P_{ij} P_{mn} \right ] \ddot{Q}_{ij} (t -d/c),
\end{equation}
where $Q_{ij}$ is the mass quadrupole moment which to leading order in $v/c$ and in Cartesian coordinates is given by
\begin{equation} \label{eqT:quad}
Q_{ij} = \int \mathrm{d}^3 x \rho(t,\mathbf{x})\left [x_i x_j - \frac{1}{3} \delta^{ij} x_l x^l \right ].
\end{equation}
$P_{ij} = \delta_{ij} -\hat{n}_i \hat{n}_j$ is the projection operator onto the plane transverse to the direction the wave is propagating,
$\hat{n}_i = x_i / \sqrt{x_j x^j}$. To find the equation describing $h_{+}$ and $h_{\times}$ for a wave propagating in the general direction
$\hat{n}$ we first consider a wave propagating along the z-axis in a coordinate system with Cartesian coordinates $(x,y,z)$ and
spherical polar coordinates $(r,\theta,\phi)$. 
If $\hat{n} = \hat{z}$ then $P_{ij}$ becomes
\begin{equation} \label{eqT:pij}
P_{ij}  = 
  \begin{pmatrix}
    1 & 0 & 0  \\
    0 & 1 & 0 \\
    0 & 0 & 0
  \end{pmatrix}_{ij},
\end{equation}
and we find that 
\begin{equation} \label{eqT:pjz}
\left [P_{im} P_{jn} - \frac{1}{2} P_{ij} P_{mn} \right ] \ddot{Q}_{ij} = 
  \begin{pmatrix}
    (\ddot{Q}_{11}-\ddot{Q}_{22})/2 & \ddot{Q}_{12} & 0  \\
    \ddot{Q}_{21} & (\ddot{Q}_{22} - \ddot{Q}_{11})/2 & 0 \\
    0 & 0 & 0
  \end{pmatrix}_{ij}.
\end{equation}
By comparing \eq{eqT:hzdir} and \eq{eqT:pjz} we see that
\begin{align} \label{eqT:zhchx}
h_{+}^{TT} &= \frac{G}{c^4 D} (\ddot{Q}_{11}-\ddot{Q}_{22}) \\ \nonumber
h_{\times}^{TT} &= \frac{2G}{c^4 D} \ddot{Q}_{12}.
\end{align}
Now consider a wave propagating in the direction given by $\hat{n'} = (\sin{\theta} \sin{\phi}, \sin{\theta} \cos{\phi},\cos{\theta})$.
We can view this as a wave propagating along the $z'$-axis of a coordinate system, with axes $(x',y',z')$, that has been constructed  
by rotating the original system about the $z$-axis by an angle $\phi$ and then about the $x$-axis by an angle $\theta$.
The rotation matrix R of the two consecutive rotations is
\begin{equation} \label{eqT:pjz}
R = 
  \begin{pmatrix}
    \cos{\phi} & \sin{\phi} & 0  \\
    -\sin{\phi} & \cos{\phi} & 0 \\
    0 & 0 & 0
  \end{pmatrix}
  \begin{pmatrix}
    0 & 0 & 0  \\
    0& \cos{\theta} & \sin{\theta} \\
    0 & -\sin{\theta} & \cos{\theta}
  \end{pmatrix}.
\end{equation}
Since $\hat{n'} = \hat{z'}$ we can use the result from \eq{eqT:zhchx}, but we have to
transform the components of $\ddot{Q}$ from the old system (x,y,z) into our new coordinate system (x',y',z').
Under the rotations described by \eq{eqT:pjz} the quadrupole moment transforms as follows
\begin{equation}
\ddot{Q}_{ij} \rightarrow \ddot{Q}_{ij}' = (R Q R^T)_{ij},
\end{equation}
where $R^T$ denotes the transposed matrix $R$.
After some straightforward, but cumbersome, algebra we arrive at
\begin{align}
\label{eqT:hp}
h_{+}^{TT} = \frac{G}{c^4 D} & \Big[ \ddot{Q}_{11} (\cos^2{\phi} - \sin^2{\phi} \cos^2{\theta})  \\ \nonumber
& + \ddot{Q}_{22} (\sin^2{\phi} - \cos^2{\phi} \cos^2{\theta}) - \ddot{Q}_{33} \sin^2{\theta} \\ \nonumber
& - \ddot{Q}_{12} (1 + \cos^2{\theta}) + \ddot{Q}_{13} \sin{\phi} \sin{2\theta} \\ \nonumber
& + \ddot{Q}_{23} \cos{\phi} \sin{2\theta} \Big]
\end{align}
and
\begin{align}
\label{eqT:hc}
h_{\times}^{TT} = \frac{G}{c^4 D} & \Big[ (\ddot{Q}_{11} - \ddot{Q}_{22}) \sin{2\phi}\cos{\theta}\\ \nonumber
& +\ddot{Q}_{12} \cos{\theta} \cos{2\phi} - \ddot{Q}_{13} \cos{\phi} \sin{\theta} \\ \nonumber
& +  2\ddot{Q}_{23} \sin{\phi} \sin{\theta} \Big]. 
\end{align}
\eq{eqT:hc} and \eq{eqT:hp} tell us is how the waveforms of the emitted GW signal depend on 
the observers location relative to the source.
If we calculate $\ddot{Q}_{ij}$ in a given coordinate system, 
when dealing with simulations it is often convenient to calculate the quadrupole moment
in the coordinate system of the simulations. We can then use \eq{eqT:hc} and \eq{eqT:hp} to compute the signal for an observer situated along the direction $\hat{n} = (\sin{\theta} \sin{\phi}, \sin{\theta} \cos{\phi},\cos{\theta})$.

In principle, we have all the ingredients we need to calculate the gravitational quadrupole radiation as observed by a distant observer 
for a slow-moving source. However, we are required to calculate the second time derivative of $Q_{ij}$. In theory this is not a problem, but
when dealing with simulations it is difficult to achieve accurate result when performing direct numerical differentiation of the
quadrupole moment. 
Furthermore, in \eq{eqT:quad} the terms $\rho x_i x_j \mathrm{d} V$ will give large weight to
slow moving low density fluid elements far away from the regions where GWs are actually produced. 
On the other hand, the second time derivative of these contributions
will be small, because faster moving and denser matter will contribute more to the GW amplitudes.  
Again, this is not problematic in theory, but when numerically computing the integral and second-order time derivatives 
the exact cancellation of the large average value and the rate of change of the quadrupole moment is hard to achieve.
We can circumvent these problems by using the Euler equations to rewrite \eq{eqT:quad}. This allows us to
eliminate the time derivatives completely and to write the quadrupole moment in terms of quantities 
which are more closely connected to the regions of the simulation where GWs are produced \citep{oohara_97,finn_89,blanchet_90}.

The standard Euler equations of a self-gravitating fluid are
\begin{subequations} 
\begin{align}
\partial_t \rho + \partial_i (\rho v^i) &= 0, \label{eqT:cont} \\  
\partial_t (\rho v^i) + \partial_j (\rho v^i v^j) & = -\partial_i p - \rho \partial_i \Phi, \label{eqT:mom} \\ 
\partial_t (\rho \varepsilon) + \partial_i (\rho \varepsilon v^i) & = -p \partial_i v^i, \label{eqT:enrg} 
\end{align}
\end{subequations}
where $p$, $v^i$, $\varepsilon$, and $\rho$ are the pressure, the velocity components, the internal energy density,
and the mass density of the fluid, respectively. The Newtonian potential is denoted by $\Phi$, and we have 
neglected any radiation back-reaction.  

Taking the first time derivative of \eq{eqT:quad} yields
\begin{align} \label{eqT:quaddt}
\frac{\mathrm{d}}{\mathrm{dt}}Q_{ij} &=  \frac{\mathrm{d}}{\mathrm{dt}} 
\int \mathrm{d}^3 x \rho\left [x_i x_j - \frac{1}{3} \delta^{ij} x_l x^l \right ] \nonumber \\ 
& = \int \mathrm{d}^3 x \left [x_i x_j - \frac{1}{3} \delta^{ij} x_l x^l \right ] \partial_t \rho \nonumber \\
& = -\int \mathrm{d}^3 x \left [x_i x_j - \frac{1}{3} \delta^{ij} x_l x^l \right ] \partial_k (\rho v^k),
\end{align}
where we used \eq{eqT:cont} in the last line to replace $\partial_t \rho$ with $-\partial_k (\rho v^k)$.
Now we integrate \eq{eqT:quaddt} by parts and use the fact that astrophysical GW sources have finite sizes 
to discard the boundary terms. We find that
\begin{align} \label{eqT:quaddt2}
\frac{\mathrm{d}}{\mathrm{dt}}Q_{ij} &= \int \mathrm{d}^3 x \rho  v^k \partial_k \left [x_i x_j - \frac{1}{3} \delta^{ij} x_l x^l \right ] \nonumber \\
&= \int \mathrm{d}^3 x \rho v^k \left  [\partial_k x_i x_j - \frac{1}{3} \delta^{ij} \partial_k (x_l x^l) \right ] \nonumber \\
&= \int \mathrm{d}^3 x \rho \left [x_i v_j + v_i x_j - \frac{2}{3} \delta^{ij} v_l x^l \right ],
\end{align}
where we integrated the first line by parts to get to the second line, and when going from the second to the third line we used the fact that
$\partial_k x_i = \delta^{ki}$. We have now removed the first time derivative from \eq{eqT:quad}, and we have replaced one
power of $x^i$ with $v^i$ which reduces the weight given to slow moving regions.
Next we take the time derivative of \eq{eqT:quaddt2}, which yields
\begin{align} \label{eqT:quaddtt}
\frac{\mathrm{d}^2}{\mathrm{dt}^2}Q_{ij} &= \frac{\mathrm{d}}{\mathrm{dt}} \int \mathrm{d}^3 x \rho \left  
[x_i v_j + v_i x_j - \frac{2}{3} \delta^{ij} v_l x^l \right ], \nonumber \\
& = \int \mathrm{d}^3 x \left [x_i (\partial_t \rho v_j) +  (\partial_t \rho v_i) x_j - \frac{2}{3} \delta^{ij} \partial_t ( \rho v_l) x^l \right ].
\end{align}
Now we use \eq{eqT:mom} to remove $\partial_t \rho v_i$ and we get
\begin{align} \label{eqT:quaddtt2}
\frac{\mathrm{d}^2}{\mathrm{dt}^2} Q_{ij} & = \int \mathrm{d}^3 x \bigg[ x_i (-\partial_j p - \rho \partial_j \Phi - \partial_k (\rho v_j v_k)) \nonumber \\
&+ x_j (-\partial_i p - \rho \partial_i \Phi - \partial_k (\rho v_i v_k)) \nonumber \\
&- \frac{2}{3} \delta^{ij} x^l (-\partial_l p - \rho \partial_l \Phi - \partial_k (\rho v_l v_k)) \bigg].
\end{align}
After integrating by parts terms of the form $\partial_i p$ and $\partial_k (\rho v_i v_k))$
we are left with
\begin{align} \label{eqT:quaddtt3}
\frac{\mathrm{d}^2}{\mathrm{dt}^2} Q_{ij} & = \int \mathrm{d}^3 x \rho \bigg[ 2 v_i v_j  - x_i \partial_j \Phi -  x_j \partial_i \Phi \nonumber \\
&- \frac{2}{3} \delta^{ij} (\rho v_l v_l - \rho x_l \partial_l \Phi)  \bigg]. 
\end{align}
Note that the terms containing the pressure in \eq{eqT:quaddtt2} cancel. The last step is to write \eq{eqT:quaddtt3} in a slightly more 
compact form    
\begin{equation} \label{eqT:STFQ}
\ddot{Q}_{ij} =\mathrm{STF} \bigg[2 \int \mathrm{d}^3 x \rho \Big( v_i v_j - x_i \partial_j \Phi \Big) \bigg].
\end{equation}
Here $\mathrm{STF}$ denotes the projection operator
onto the symmetric trace-free part, $\mathrm{STF}[A_{ij}] = \frac{1}{2}A_{ij} + \frac{1}{2}A_{ji} - \frac{1}{3} \delta_{ij} A_{ll} $.
In the end, we are left with an expression for the second time derivative of the quadrupole moment that
depends on first-order spatial derivatives of the gravitational potential and the Cartesian velocity components.
These terms have larger numerical values in the regions of the simulation generating most of the GWs, which is an advantage when numerically calculating the integral in \eq{eqT:STFQ}. We have also 
eliminated the troublesome second-order time derivatives and arrived at a formula that is much more suited for 
computing waveforms in numerical studies than our starting point.

Note that in axisymmetry the only independent component of $\munu{h}^{TT}$ is 
\begin{equation}
\mathbf{h}^{TT}_{\theta \theta} = \frac{1}{8}\sqrt{\frac{15}{\pi}} \sin^2{\theta} \frac{A_{20}^\mathrm{E2}}{D},
\end{equation}
where $D$ is the distance to the source, $\theta$ is the inclination angle of the observer with respect to the
axis of symmetry, and $A_{20}^\mathrm{E2}$ represents the only non-zero quadrupole amplitude.
In spherical coordinates $A_{20}^\mathrm{E2}$ can be expressed as follows
\begin{eqnarray} \label{eq:2dquad}
A_{20}^\mathrm{E2} (t) =  \frac{G}{c^4} \frac{16 \pi^{3/2}}{\sqrt{15}} \int_{-1}^{1}\int^{\infty}_0 \rho \left [ v_r^2(3 z^2 - 1)+ \right. \nonumber \\
v_{\theta}^2(2-3 z^2) - v_{\phi}^2 - 6 v_r v_{\theta} z\sqrt{1-z^2} + \nonumber \\
r \partial_r \Phi (3 z^2 - 1) +\left. 3 \partial_{\theta}z\sqrt{1-z^2} \Phi \right ]r^2 dr \, dz.
\end{eqnarray}
Here, $v_i$ and $\partial_i$ ($i = r, \theta, \phi$) represent the velocity components and derivatives, respectively, along
the basis vectors of the spherical coordinate system, and $z \equiv \cos \theta$.
For details we refer the reader to \cite{mueller_97}.

In the quadrupole framework the energy, $E$, radiated by GWs is given by 
\begin{align}
E &=\frac{G}{5 c^5} \int \mathrm{d}t \, \dddot{Q}_{ij} \dddot{Q}_{ij},
\end{align}
and the spectral energy density of the GWs is given by 
\begin{align}
\frac{\mathrm{d} E}{\mathrm{d} f} &= \frac{2G}{5 c^5} (2\pi f)^2 \int \mathrm{d}t \widetilde{\ddot{Q}}_{ij} \widetilde{\ddot{Q}}_{ij},
\end{align}
where a tilde denotes a Fourier transform, and $f$ is the
frequency. We define the Fourier transform as follows:
\begin{equation}
\widetilde{\ddot{Q}}_{ij}(f) = \int_{-\infty}^{\infty} \ddot{Q}_{ij}(t) e^{-2 \pi i f t} \ud t. 
\end{equation}
For a discrete time series the Fourier transform is replaced by the discrete Fourier transform (DFT).
We define the DFT, $\widetilde{X}_k$, as follows: 
\begin{equation} \label{eq:DFT}
\widetilde{X}_k (f_k) = \frac{1}{M}  \sum^M_{m=1} x_m e^{-2\pi i k m/M},
\end{equation}
Here, $x_m$ is the time series obtained by sampling the underlying continuous signal at $M$ discrete times. 
$f_k = k/T$ is the frequency of bin $k$, where $T$ is the duration of the signal.

We will repeat these definitions and discuss our Fourier analysis in detail in a later chapter. 